\documentclass[a4paper,10pt]{article}
%\documentclass[a4paper,10pt]{scrartcl}

\usepackage[utf8]{inputenc}
\usepackage[margin=1in]{geometry}
\usepackage{tikz}
\usepackage{alltt}
\usepackage{fancyvrb}
\newcommand{\mytilde}{$\sim$}


\usepackage{float}
\pdfinfo{%
  /Title    (Custom Computing: Assessed Coursework)
  /Author   (Ioannis Kassinopoulos)
  /Creator  (Ioannis Kassinopoulos)
  /Producer (Ioannis Kassinopoulos)
  /Subject  (Custom Computing: Assessed Coursework)
  /Keywords (custom,computing,coursework,imperial)
}
\begin{document}

\title{Custom Computing: Assessed Coursework}
\author{Ioannis Kassinopoulos}
\date{\today}
\maketitle
\section*{Question 1}
\textbf{Recurring engineering costs} are the costs that will occur in a repeating fashion during the production, usually involving fabriction.
These costs are usually descriped in a per unit form.
\\[0.25cm]
\textbf{Non-recurring engineering cost} is the one-time up-front cost 
for research, design, testing and development of a new product.
\\[0.25cm]
As we can see below, the minimum number of units that need to be sold for the ASIC implementation to be cost-effective is 1 million units.
\\[0.25cm]
$C_{FPGA} > C_{ASIC}$
$\Rightarrow \pounds 0 + \pounds 2 \times N_{units} > \pounds 10^6 + \pounds 1 \times N_{units} $ 
$\Rightarrow N_{units} > 10^6$

\section*{Question 2}
\subsection*{(a) Diagramatic and symbolic Simulation}
\subsubsection*{Diagram of circuit Q1}
\begin{figure}[!h]
\begin{center}
% Graphic for TeX using PGF
% Title: /home/yiannis/Diagram1.dia
% Creator: Dia v0.97.2
% CreationDate: Thu Feb 28 16:05:15 2013
% For: yiannis
% \usepackage{tikz}
% The following commands are not supported in PSTricks at present
% We define them conditionally, so when they are implemented,
% this pgf file will use them.
\ifx\du\undefined
  \newlength{\du}
\fi
\setlength{\du}{15\unitlength}
\begin{tikzpicture}[thick, scale=0.3]
\pgftransformxscale{1.000000}
\pgftransformyscale{-1.000000}
\definecolor{dialinecolor}{rgb}{0.000000, 0.000000, 0.000000}
\pgfsetstrokecolor{dialinecolor}
\definecolor{dialinecolor}{rgb}{1.000000, 1.000000, 1.000000}
\pgfsetfillcolor{dialinecolor}
\definecolor{dialinecolor}{rgb}{1.000000, 1.000000, 1.000000}
\pgfsetfillcolor{dialinecolor}
\fill (2.541048\du,-117.295272\du)--(2.541048\du,-110.044645\du)--(9.801177\du,-110.044645\du)--(9.801177\du,-117.295272\du)--cycle;
\pgfsetlinewidth{0.100000\du}
\pgfsetdash{}{0pt}
\pgfsetdash{}{0pt}
\pgfsetmiterjoin
\definecolor{dialinecolor}{rgb}{0.000000, 0.000000, 0.000000}
\pgfsetstrokecolor{dialinecolor}
\draw (2.541048\du,-117.295272\du)--(2.541048\du,-110.044645\du)--(9.801177\du,-110.044645\du)--(9.801177\du,-117.295272\du)--cycle;
% setfont left to latex
\definecolor{dialinecolor}{rgb}{0.000000, 0.000000, 0.000000}
\pgfsetstrokecolor{dialinecolor}
\node at (6.171113\du,-113.474958\du){ADD};
\definecolor{dialinecolor}{rgb}{1.000000, 1.000000, 1.000000}
\pgfsetfillcolor{dialinecolor}
\fill (2.535908\du,-127.308673\du)--(2.535908\du,-120.058046\du)--(9.796037\du,-120.058046\du)--(9.796037\du,-127.308673\du)--cycle;
\pgfsetlinewidth{0.100000\du}
\pgfsetdash{}{0pt}
\pgfsetdash{}{0pt}
\pgfsetmiterjoin
\definecolor{dialinecolor}{rgb}{0.000000, 0.000000, 0.000000}
\pgfsetstrokecolor{dialinecolor}
\draw (2.535908\du,-127.308673\du)--(2.535908\du,-120.058046\du)--(9.796037\du,-120.058046\du)--(9.796037\du,-127.308673\du)--cycle;
% setfont left to latex
\definecolor{dialinecolor}{rgb}{0.000000, 0.000000, 0.000000}
\pgfsetstrokecolor{dialinecolor}
\node at (6.165972\du,-123.488360\du){D-1};
% setfont left to latex
\definecolor{dialinecolor}{rgb}{0.000000, 0.000000, 0.000000}
\pgfsetstrokecolor{dialinecolor}
\node[anchor=west] at (5.384819\du,-123.499559\du){};
\pgfsetlinewidth{0.100000\du}
\pgfsetdash{}{0pt}
\pgfsetdash{}{0pt}
\pgfsetmiterjoin
\pgfsetbuttcap
{
\definecolor{dialinecolor}{rgb}{0.000000, 0.000000, 0.000000}
\pgfsetfillcolor{dialinecolor}
% was here!!!
{\pgfsetcornersarced{\pgfpoint{0.000000\du}{0.000000\du}}\definecolor{dialinecolor}{rgb}{0.000000, 0.000000, 0.000000}
\pgfsetstrokecolor{dialinecolor}
\draw (2.535908\du,-123.683360\du)--(-15.149771\du,-123.691849\du);
}}
\pgfsetlinewidth{0.100000\du}
\pgfsetdash{}{0pt}
\pgfsetdash{}{0pt}
\pgfsetbuttcap
{
\definecolor{dialinecolor}{rgb}{0.000000, 0.000000, 0.000000}
\pgfsetfillcolor{dialinecolor}
% was here!!!
\definecolor{dialinecolor}{rgb}{0.000000, 0.000000, 0.000000}
\pgfsetstrokecolor{dialinecolor}
\draw (2.541048\du,-111.857302\du)--(-15.149771\du,-111.836702\du);
}
\pgfsetlinewidth{0.100000\du}
\pgfsetdash{}{0pt}
\pgfsetdash{}{0pt}
\pgfsetmiterjoin
\pgfsetbuttcap
{
\definecolor{dialinecolor}{rgb}{0.000000, 0.000000, 0.000000}
\pgfsetfillcolor{dialinecolor}
% was here!!!
{\pgfsetcornersarced{\pgfpoint{0.000000\du}{0.000000\du}}\definecolor{dialinecolor}{rgb}{0.000000, 0.000000, 0.000000}
\pgfsetstrokecolor{dialinecolor}
\draw (2.541048\du,-115.482615\du)--(-11.106155\du,-115.482615\du)--(-11.106155\du,-123.645899\du)--(-11.152106\du,-123.645899\du);
}}
\pgfsetlinewidth{0.100000\du}
\pgfsetdash{}{0pt}
\pgfsetdash{}{0pt}
\pgfsetmiterjoin
\pgfsetbuttcap
{
\definecolor{dialinecolor}{rgb}{0.000000, 0.000000, 0.000000}
\pgfsetfillcolor{dialinecolor}
% was here!!!
{\pgfsetcornersarced{\pgfpoint{0.000000\du}{0.000000\du}}\definecolor{dialinecolor}{rgb}{0.000000, 0.000000, 0.000000}
\pgfsetstrokecolor{dialinecolor}
\draw (9.796037\du,-123.683360\du)--(15.177348\du,-123.737799\du);
}}
\pgfsetlinewidth{0.100000\du}
\pgfsetdash{}{0pt}
\pgfsetdash{}{0pt}
\pgfsetmiterjoin
\pgfsetbuttcap
{
\definecolor{dialinecolor}{rgb}{0.000000, 0.000000, 0.000000}
\pgfsetfillcolor{dialinecolor}
% was here!!!
{\pgfsetcornersarced{\pgfpoint{0.000000\du}{0.000000\du}}\definecolor{dialinecolor}{rgb}{0.000000, 0.000000, 0.000000}
\pgfsetstrokecolor{dialinecolor}
\draw (9.939028\du,-113.536859\du)--(15.085448\du,-113.536859\du);
}}
% setfont left to latex
\definecolor{dialinecolor}{rgb}{0.000000, 0.000000, 0.000000}
\pgfsetstrokecolor{dialinecolor}
\node[anchor=west] at (6.171113\du,-113.669958\du){};
% setfont left to latex
\definecolor{dialinecolor}{rgb}{0.000000, 0.000000, 0.000000}
\pgfsetstrokecolor{dialinecolor}
\node[anchor=west] at (6.165972\du,-123.683360\du){};
% setfont left to latex
\definecolor{dialinecolor}{rgb}{0.000000, 0.000000, 0.000000}
\pgfsetstrokecolor{dialinecolor}
\node[anchor=west] at (6.165972\du,-123.683360\du){};
\end{tikzpicture}

\caption{the circuit as derrived from Q1}
\end{center}
\end{figure}
\subsubsection*{Diagram of circuit P1}
\begin{figure}[!h]
\begin{center}
% Graphic for TeX using PGF
% Title: /home/yiannis/Desktop/cw-custom/diafiles/p1q1.dia
% Creator: Dia v0.97.2
% CreationDate: Thu Feb 28 16:59:01 2013
% For: yiannis
% \usepackage{tikz}
% The following commands are not supported in PSTricks at present
% We define them conditionally, so when they are implemented,
% this pgf file will use them.
\ifx\du\undefined
  \newlength{\du}
\fi
\setlength{\du}{15\unitlength}
\begin{tikzpicture}[thick, scale=0.26]
\pgftransformxscale{1.000000}
\pgftransformyscale{-1.000000}
\definecolor{dialinecolor}{rgb}{0.000000, 0.000000, 0.000000}
\pgfsetstrokecolor{dialinecolor}
\definecolor{dialinecolor}{rgb}{1.000000, 1.000000, 1.000000}
\pgfsetfillcolor{dialinecolor}
\definecolor{dialinecolor}{rgb}{1.000000, 1.000000, 1.000000}
\pgfsetfillcolor{dialinecolor}
\fill (14.240994\du,-115.995006\du)--(14.240994\du,-108.744379\du)--(21.501123\du,-108.744379\du)--(21.501123\du,-115.995006\du)--cycle;
\pgfsetlinewidth{0.100000\du}
\pgfsetdash{}{0pt}
\pgfsetdash{}{0pt}
\pgfsetmiterjoin
\definecolor{dialinecolor}{rgb}{0.000000, 0.000000, 0.000000}
\pgfsetstrokecolor{dialinecolor}
\draw (14.240994\du,-115.995006\du)--(14.240994\du,-108.744379\du)--(21.501123\du,-108.744379\du)--(21.501123\du,-115.995006\du)--cycle;
% setfont left to latex
\definecolor{dialinecolor}{rgb}{0.000000, 0.000000, 0.000000}
\pgfsetstrokecolor{dialinecolor}
\node at (17.871058\du,-112.197471\du){ADD};
\definecolor{dialinecolor}{rgb}{1.000000, 1.000000, 1.000000}
\pgfsetfillcolor{dialinecolor}
\fill (14.235854\du,-126.009006\du)--(14.235854\du,-118.758379\du)--(21.495983\du,-118.758379\du)--(21.495983\du,-126.009006\du)--cycle;
\pgfsetlinewidth{0.100000\du}
\pgfsetdash{}{0pt}
\pgfsetdash{}{0pt}
\pgfsetmiterjoin
\definecolor{dialinecolor}{rgb}{0.000000, 0.000000, 0.000000}
\pgfsetstrokecolor{dialinecolor}
\draw (14.235854\du,-126.009006\du)--(14.235854\du,-118.758379\du)--(21.495983\du,-118.758379\du)--(21.495983\du,-126.009006\du)--cycle;
% setfont left to latex
\definecolor{dialinecolor}{rgb}{0.000000, 0.000000, 0.000000}
\pgfsetstrokecolor{dialinecolor}
\node at (17.865918\du,-122.211471\du){D-1};
% setfont left to latex
\definecolor{dialinecolor}{rgb}{0.000000, 0.000000, 0.000000}
\pgfsetstrokecolor{dialinecolor}
\node[anchor=west] at (17.084764\du,-122.200006\du){};
\pgfsetlinewidth{0.100000\du}
\pgfsetdash{}{0pt}
\pgfsetdash{}{0pt}
\pgfsetmiterjoin
\pgfsetbuttcap
{
\definecolor{dialinecolor}{rgb}{0.000000, 0.000000, 0.000000}
\pgfsetfillcolor{dialinecolor}
% was here!!!
\pgfsetarrowsend{to}
{\pgfsetcornersarced{\pgfpoint{0.000000\du}{0.000000\du}}\definecolor{dialinecolor}{rgb}{0.000000, 0.000000, 0.000000}
\pgfsetstrokecolor{dialinecolor}
\draw (14.235854\du,-122.383693\du)--(-3.449856\du,-122.392006\du);
}}
\pgfsetlinewidth{0.100000\du}
\pgfsetdash{}{0pt}
\pgfsetdash{}{0pt}
\pgfsetbuttcap
{
\definecolor{dialinecolor}{rgb}{0.000000, 0.000000, 0.000000}
\pgfsetfillcolor{dialinecolor}
% was here!!!
\pgfsetarrowsstart{to}
\definecolor{dialinecolor}{rgb}{0.000000, 0.000000, 0.000000}
\pgfsetstrokecolor{dialinecolor}
\draw (14.240994\du,-110.557036\du)--(-3.449856\du,-110.537006\du);
}
\pgfsetlinewidth{0.100000\du}
\pgfsetdash{}{0pt}
\pgfsetdash{}{0pt}
\pgfsetmiterjoin
\pgfsetbuttcap
{
\definecolor{dialinecolor}{rgb}{0.000000, 0.000000, 0.000000}
\pgfsetfillcolor{dialinecolor}
% was here!!!
\pgfsetarrowsstart{to}
{\pgfsetcornersarced{\pgfpoint{0.000000\du}{0.000000\du}}\definecolor{dialinecolor}{rgb}{0.000000, 0.000000, 0.000000}
\pgfsetstrokecolor{dialinecolor}
\draw (14.240994\du,-114.182349\du)--(0.593744\du,-114.182349\du)--(0.593744\du,-122.346006\du)--(0.547844\du,-122.346006\du);
}}
% setfont left to latex
\definecolor{dialinecolor}{rgb}{0.000000, 0.000000, 0.000000}
\pgfsetstrokecolor{dialinecolor}
\node[anchor=west] at (17.871058\du,-112.369693\du){};
% setfont left to latex
\definecolor{dialinecolor}{rgb}{0.000000, 0.000000, 0.000000}
\pgfsetstrokecolor{dialinecolor}
\node[anchor=west] at (17.865918\du,-122.383693\du){};
% setfont left to latex
\definecolor{dialinecolor}{rgb}{0.000000, 0.000000, 0.000000}
\pgfsetstrokecolor{dialinecolor}
\node[anchor=west] at (17.865918\du,-122.383693\du){};
\definecolor{dialinecolor}{rgb}{1.000000, 1.000000, 1.000000}
\pgfsetfillcolor{dialinecolor}
\fill (39.080876\du,-115.993845\du)--(39.080876\du,-108.743218\du)--(46.341004\du,-108.743218\du)--(46.341004\du,-115.993845\du)--cycle;
\pgfsetlinewidth{0.100000\du}
\pgfsetdash{}{0pt}
\pgfsetdash{}{0pt}
\pgfsetmiterjoin
\definecolor{dialinecolor}{rgb}{0.000000, 0.000000, 0.000000}
\pgfsetstrokecolor{dialinecolor}
\draw (39.080876\du,-115.993845\du)--(39.080876\du,-108.743218\du)--(46.341004\du,-108.743218\du)--(46.341004\du,-115.993845\du)--cycle;
% setfont left to latex
\definecolor{dialinecolor}{rgb}{0.000000, 0.000000, 0.000000}
\pgfsetstrokecolor{dialinecolor}
\node at (42.710940\du,-112.196309\du){ADD};
\definecolor{dialinecolor}{rgb}{1.000000, 1.000000, 1.000000}
\pgfsetfillcolor{dialinecolor}
\fill (39.075736\du,-126.007845\du)--(39.075736\du,-118.757218\du)--(46.335864\du,-118.757218\du)--(46.335864\du,-126.007845\du)--cycle;
\pgfsetlinewidth{0.100000\du}
\pgfsetdash{}{0pt}
\pgfsetdash{}{0pt}
\pgfsetmiterjoin
\definecolor{dialinecolor}{rgb}{0.000000, 0.000000, 0.000000}
\pgfsetstrokecolor{dialinecolor}
\draw (39.075736\du,-126.007845\du)--(39.075736\du,-118.757218\du)--(46.335864\du,-118.757218\du)--(46.335864\du,-126.007845\du)--cycle;
% setfont left to latex
\definecolor{dialinecolor}{rgb}{0.000000, 0.000000, 0.000000}
\pgfsetstrokecolor{dialinecolor}
\node at (42.705800\du,-122.210309\du){D-1};
% setfont left to latex
\definecolor{dialinecolor}{rgb}{0.000000, 0.000000, 0.000000}
\pgfsetstrokecolor{dialinecolor}
\node[anchor=west] at (41.924646\du,-122.198845\du){};
\pgfsetlinewidth{0.100000\du}
\pgfsetdash{}{0pt}
\pgfsetdash{}{0pt}
\pgfsetmiterjoin
\pgfsetbuttcap
{
\definecolor{dialinecolor}{rgb}{0.000000, 0.000000, 0.000000}
\pgfsetfillcolor{dialinecolor}
% was here!!!
\pgfsetarrowsend{to}
{\pgfsetcornersarced{\pgfpoint{0.000000\du}{0.000000\du}}\definecolor{dialinecolor}{rgb}{0.000000, 0.000000, 0.000000}
\pgfsetstrokecolor{dialinecolor}
\draw (39.075736\du,-122.382532\du)--(21.495983\du,-122.383693\du);
}}
\pgfsetlinewidth{0.100000\du}
\pgfsetdash{}{0pt}
\pgfsetdash{}{0pt}
\pgfsetbuttcap
{
\definecolor{dialinecolor}{rgb}{0.000000, 0.000000, 0.000000}
\pgfsetfillcolor{dialinecolor}
% was here!!!
\pgfsetarrowsstart{to}
\definecolor{dialinecolor}{rgb}{0.000000, 0.000000, 0.000000}
\pgfsetstrokecolor{dialinecolor}
\draw (39.080876\du,-110.555875\du)--(21.501123\du,-110.557036\du);
}
\pgfsetlinewidth{0.100000\du}
\pgfsetdash{}{0pt}
\pgfsetdash{}{0pt}
\pgfsetmiterjoin
\pgfsetbuttcap
{
\definecolor{dialinecolor}{rgb}{0.000000, 0.000000, 0.000000}
\pgfsetfillcolor{dialinecolor}
% was here!!!
\pgfsetarrowsstart{to}
{\pgfsetcornersarced{\pgfpoint{0.000000\du}{0.000000\du}}\definecolor{dialinecolor}{rgb}{0.000000, 0.000000, 0.000000}
\pgfsetstrokecolor{dialinecolor}
\draw (39.080876\du,-114.181188\du)--(25.433626\du,-114.181188\du)--(25.433626\du,-122.344845\du)--(25.387726\du,-122.344845\du);
}}
% setfont left to latex
\definecolor{dialinecolor}{rgb}{0.000000, 0.000000, 0.000000}
\pgfsetstrokecolor{dialinecolor}
\node[anchor=west] at (42.710940\du,-112.368532\du){};
% setfont left to latex
\definecolor{dialinecolor}{rgb}{0.000000, 0.000000, 0.000000}
\pgfsetstrokecolor{dialinecolor}
\node[anchor=west] at (42.705800\du,-122.382532\du){};
% setfont left to latex
\definecolor{dialinecolor}{rgb}{0.000000, 0.000000, 0.000000}
\pgfsetstrokecolor{dialinecolor}
\node[anchor=west] at (42.705800\du,-122.382532\du){};
\definecolor{dialinecolor}{rgb}{1.000000, 1.000000, 1.000000}
\pgfsetfillcolor{dialinecolor}
\fill (63.880875\du,-115.984880\du)--(63.880875\du,-108.734253\du)--(71.141004\du,-108.734253\du)--(71.141004\du,-115.984880\du)--cycle;
\pgfsetlinewidth{0.100000\du}
\pgfsetdash{}{0pt}
\pgfsetdash{}{0pt}
\pgfsetmiterjoin
\definecolor{dialinecolor}{rgb}{0.000000, 0.000000, 0.000000}
\pgfsetstrokecolor{dialinecolor}
\draw (63.880875\du,-115.984880\du)--(63.880875\du,-108.734253\du)--(71.141004\du,-108.734253\du)--(71.141004\du,-115.984880\du)--cycle;
% setfont left to latex
\definecolor{dialinecolor}{rgb}{0.000000, 0.000000, 0.000000}
\pgfsetstrokecolor{dialinecolor}
\node at (67.510940\du,-112.187345\du){ADD};
\definecolor{dialinecolor}{rgb}{1.000000, 1.000000, 1.000000}
\pgfsetfillcolor{dialinecolor}
\fill (63.875735\du,-125.998880\du)--(63.875735\du,-118.748253\du)--(71.135864\du,-118.748253\du)--(71.135864\du,-125.998880\du)--cycle;
\pgfsetlinewidth{0.100000\du}
\pgfsetdash{}{0pt}
\pgfsetdash{}{0pt}
\pgfsetmiterjoin
\definecolor{dialinecolor}{rgb}{0.000000, 0.000000, 0.000000}
\pgfsetstrokecolor{dialinecolor}
\draw (63.875735\du,-125.998880\du)--(63.875735\du,-118.748253\du)--(71.135864\du,-118.748253\du)--(71.135864\du,-125.998880\du)--cycle;
% setfont left to latex
\definecolor{dialinecolor}{rgb}{0.000000, 0.000000, 0.000000}
\pgfsetstrokecolor{dialinecolor}
\node at (67.505800\du,-122.201345\du){D-1};
% setfont left to latex
\definecolor{dialinecolor}{rgb}{0.000000, 0.000000, 0.000000}
\pgfsetstrokecolor{dialinecolor}
\node[anchor=west] at (66.724645\du,-122.189880\du){};
\pgfsetlinewidth{0.100000\du}
\pgfsetdash{}{0pt}
\pgfsetdash{}{0pt}
\pgfsetmiterjoin
\pgfsetbuttcap
{
\definecolor{dialinecolor}{rgb}{0.000000, 0.000000, 0.000000}
\pgfsetfillcolor{dialinecolor}
% was here!!!
\pgfsetarrowsend{to}
{\pgfsetcornersarced{\pgfpoint{0.000000\du}{0.000000\du}}\definecolor{dialinecolor}{rgb}{0.000000, 0.000000, 0.000000}
\pgfsetstrokecolor{dialinecolor}
\draw (63.875735\du,-122.373567\du)--(46.335864\du,-122.382532\du);
}}
\pgfsetlinewidth{0.100000\du}
\pgfsetdash{}{0pt}
\pgfsetdash{}{0pt}
\pgfsetbuttcap
{
\definecolor{dialinecolor}{rgb}{0.000000, 0.000000, 0.000000}
\pgfsetfillcolor{dialinecolor}
% was here!!!
\pgfsetarrowsstart{to}
\definecolor{dialinecolor}{rgb}{0.000000, 0.000000, 0.000000}
\pgfsetstrokecolor{dialinecolor}
\draw (63.880875\du,-110.546910\du)--(46.341004\du,-110.555875\du);
}
\pgfsetlinewidth{0.100000\du}
\pgfsetdash{}{0pt}
\pgfsetdash{}{0pt}
\pgfsetmiterjoin
\pgfsetbuttcap
{
\definecolor{dialinecolor}{rgb}{0.000000, 0.000000, 0.000000}
\pgfsetfillcolor{dialinecolor}
% was here!!!
\pgfsetarrowsstart{to}
{\pgfsetcornersarced{\pgfpoint{0.000000\du}{0.000000\du}}\definecolor{dialinecolor}{rgb}{0.000000, 0.000000, 0.000000}
\pgfsetstrokecolor{dialinecolor}
\draw (63.880875\du,-114.172224\du)--(50.233625\du,-114.172224\du)--(50.233625\du,-122.335880\du)--(50.187725\du,-122.335880\du);
}}
% setfont left to latex
\definecolor{dialinecolor}{rgb}{0.000000, 0.000000, 0.000000}
\pgfsetstrokecolor{dialinecolor}
\node[anchor=west] at (67.510940\du,-112.359567\du){};
% setfont left to latex
\definecolor{dialinecolor}{rgb}{0.000000, 0.000000, 0.000000}
\pgfsetstrokecolor{dialinecolor}
\node[anchor=west] at (67.505800\du,-122.373567\du){};
% setfont left to latex
\definecolor{dialinecolor}{rgb}{0.000000, 0.000000, 0.000000}
\pgfsetstrokecolor{dialinecolor}
\node[anchor=west] at (67.505800\du,-122.373567\du){};
\definecolor{dialinecolor}{rgb}{1.000000, 1.000000, 1.000000}
\pgfsetfillcolor{dialinecolor}
\fill (88.728478\du,-115.987915\du)--(88.728478\du,-108.737289\du)--(95.988607\du,-108.737289\du)--(95.988607\du,-115.987915\du)--cycle;
\pgfsetlinewidth{0.100000\du}
\pgfsetdash{}{0pt}
\pgfsetdash{}{0pt}
\pgfsetmiterjoin
\definecolor{dialinecolor}{rgb}{0.000000, 0.000000, 0.000000}
\pgfsetstrokecolor{dialinecolor}
\draw (88.728478\du,-115.987915\du)--(88.728478\du,-108.737289\du)--(95.988607\du,-108.737289\du)--(95.988607\du,-115.987915\du)--cycle;
% setfont left to latex
\definecolor{dialinecolor}{rgb}{0.000000, 0.000000, 0.000000}
\pgfsetstrokecolor{dialinecolor}
\node at (92.358542\du,-112.190380\du){ADD};
\definecolor{dialinecolor}{rgb}{1.000000, 1.000000, 1.000000}
\pgfsetfillcolor{dialinecolor}
\fill (88.723338\du,-126.001915\du)--(88.723338\du,-118.751289\du)--(95.983467\du,-118.751289\du)--(95.983467\du,-126.001915\du)--cycle;
\pgfsetlinewidth{0.100000\du}
\pgfsetdash{}{0pt}
\pgfsetdash{}{0pt}
\pgfsetmiterjoin
\definecolor{dialinecolor}{rgb}{0.000000, 0.000000, 0.000000}
\pgfsetstrokecolor{dialinecolor}
\draw (88.723338\du,-126.001915\du)--(88.723338\du,-118.751289\du)--(95.983467\du,-118.751289\du)--(95.983467\du,-126.001915\du)--cycle;
% setfont left to latex
\definecolor{dialinecolor}{rgb}{0.000000, 0.000000, 0.000000}
\pgfsetstrokecolor{dialinecolor}
\node at (92.353402\du,-122.204380\du){D-1};
% setfont left to latex
\definecolor{dialinecolor}{rgb}{0.000000, 0.000000, 0.000000}
\pgfsetstrokecolor{dialinecolor}
\node[anchor=west] at (91.572248\du,-122.192915\du){};
\pgfsetlinewidth{0.100000\du}
\pgfsetdash{}{0pt}
\pgfsetdash{}{0pt}
\pgfsetmiterjoin
\pgfsetbuttcap
{
\definecolor{dialinecolor}{rgb}{0.000000, 0.000000, 0.000000}
\pgfsetfillcolor{dialinecolor}
% was here!!!
\pgfsetarrowsend{to}
{\pgfsetcornersarced{\pgfpoint{0.000000\du}{0.000000\du}}\definecolor{dialinecolor}{rgb}{0.000000, 0.000000, 0.000000}
\pgfsetstrokecolor{dialinecolor}
\draw (88.723338\du,-122.376602\du)--(71.135864\du,-122.373567\du);
}}
\pgfsetlinewidth{0.100000\du}
\pgfsetdash{}{0pt}
\pgfsetdash{}{0pt}
\pgfsetbuttcap
{
\definecolor{dialinecolor}{rgb}{0.000000, 0.000000, 0.000000}
\pgfsetfillcolor{dialinecolor}
% was here!!!
\pgfsetarrowsstart{to}
\definecolor{dialinecolor}{rgb}{0.000000, 0.000000, 0.000000}
\pgfsetstrokecolor{dialinecolor}
\draw (88.728478\du,-110.549945\du)--(71.141004\du,-110.546910\du);
}
\pgfsetlinewidth{0.100000\du}
\pgfsetdash{}{0pt}
\pgfsetdash{}{0pt}
\pgfsetmiterjoin
\pgfsetbuttcap
{
\definecolor{dialinecolor}{rgb}{0.000000, 0.000000, 0.000000}
\pgfsetfillcolor{dialinecolor}
% was here!!!
\pgfsetarrowsstart{to}
{\pgfsetcornersarced{\pgfpoint{0.000000\du}{0.000000\du}}\definecolor{dialinecolor}{rgb}{0.000000, 0.000000, 0.000000}
\pgfsetstrokecolor{dialinecolor}
\draw (88.728478\du,-114.175259\du)--(75.081228\du,-114.175259\du)--(75.081228\du,-122.338915\du)--(75.035328\du,-122.338915\du);
}}
% setfont left to latex
\definecolor{dialinecolor}{rgb}{0.000000, 0.000000, 0.000000}
\pgfsetstrokecolor{dialinecolor}
\node[anchor=west] at (92.358542\du,-112.362602\du){};
% setfont left to latex
\definecolor{dialinecolor}{rgb}{0.000000, 0.000000, 0.000000}
\pgfsetstrokecolor{dialinecolor}
\node[anchor=west] at (92.353402\du,-122.376602\du){};
% setfont left to latex
\definecolor{dialinecolor}{rgb}{0.000000, 0.000000, 0.000000}
\pgfsetstrokecolor{dialinecolor}
\node[anchor=west] at (92.353402\du,-122.376602\du){};
\pgfsetlinewidth{0.100000\du}
\pgfsetdash{}{0pt}
\pgfsetdash{}{0pt}
\pgfsetbuttcap
{
\definecolor{dialinecolor}{rgb}{0.000000, 0.000000, 0.000000}
\pgfsetfillcolor{dialinecolor}
% was here!!!
\pgfsetarrowsend{to}
\definecolor{dialinecolor}{rgb}{0.000000, 0.000000, 0.000000}
\pgfsetstrokecolor{dialinecolor}
\pgfpathmoveto{\pgfpoint{95.988211\du}{-112.362599\du}}
\pgfpatharc{90}{-89}{5.007175\du and 5.007175\du}
\pgfusepath{stroke}
}
\pgfsetlinewidth{0.100000\du}
\pgfsetdash{}{0pt}
\pgfsetdash{}{0pt}
\pgfsetbuttcap
{
\definecolor{dialinecolor}{rgb}{0.000000, 0.000000, 0.000000}
\pgfsetfillcolor{dialinecolor}
% was here!!!
\pgfsetarrowsend{to}
\definecolor{dialinecolor}{rgb}{0.000000, 0.000000, 0.000000}
\pgfsetstrokecolor{dialinecolor}
\draw (100.951252\du,-117.555925\du)--(106.351178\du,-117.555925\du);
}
\end{tikzpicture}

\caption{the circuit as derrived from P1}
\end{center}
\end{figure}
\subsubsection*{Simulation}
The source code of the simulation (uninitialized delay) is the following:
\begin{Verbatim}[commandchars=\\\{\}]
INCLUDE "prelude.rby".
P1 n = Q1^n; fork^\mytilde1 .
Q1 = snd fork; rsh; [add,D^\mytilde1].
current = P1 4.
\end{Verbatim}
The result after executing \verb|re "a;b;c"|
\begin{Verbatim}[commandchars=\\\{\}]
0 - <a,?> ~ ((((a + ?) + ?) + ?) + ?)
1 - <b,?> ~ ((((b + ?) + ?) + ?) + ((((a + ?) + ?) + ?) + ?))
2 - <c,?> ~ ((((c + ?) + ?) + ((((a + ?) + ?) + ?) + ?)) + ((((b + ?) + ?) + ?) 
	  + ((((a + ?) + ?) + ?) + ?)))
\end{Verbatim}
The source code of the simulation (initialized delay with 0) is the following:
\begin{Verbatim}[commandchars=\\\{\}]
INCLUDE "prelude.rby".
P1 n = Q1^n; fork^\mytilde1 .
Q1 = snd fork; rsh; [add,DI 0^\mytilde1].
current = P1 4.
\end{Verbatim}
The result after executing \verb|re "a;b;c"|
\begin{Verbatim}[commandchars=\\\{\}]
0 - <a,0> ~ ((((a + 0) + 0) + 0) + 0)
1 - <b,0> ~ ((((b + 0) + 0) + 0) + ((((a + 0) + 0) + 0) + 0))
2 - <c,0> ~ ((((c + 0) + 0) + ((((a + 0) + 0) + 0) + 0)) + ((
	    ((b + 0) + 0) + 0) + ((((a + 0) + 0) + 0) + 0)))
\end{Verbatim}
The result after executing \verb|re -s 4 1"|
\begin{Verbatim}[commandchars=\\\{\}]
0 - <1,0> ~ 1
1 - <1,0> ~ 2
2 - <1,0> ~ 4
3 - <1,0> ~ 8
\end{Verbatim}
The simulation output (for 4 cycles) can be found in the included zip file.
\subsection*{(b) Use of the transformation}
We first add D registers between the Q1 series and $fork^{-1}$ which give us: $P2=Q1^n;[D,D]^n;fork^{-1}$\\[0.25cm] 
Using the transformation: $P2=(Q1;[D,D])^n;fork^{-1}$ since $Q1;[D,D]$ is equivalent to $[D,D];Q1$.\\[0.25cm] 
Now we can consider Q2 to be equal to $Q1;[D,D]$ and therefore $P2 = Q2^n;fork^{-1}$ is our definition as stated by the question.\\[0.25cm] 
Going deeper into the definition, we now need to check if this means that D registers have been added between the adders. \\[0.25cm] 
$Q2 = Q1;[D,D] $\\[0.25cm]
$\Rightarrow Q2 = snd\: fork;rsh;[add,D^{-1}];[D,D]$ \\[0.25cm]
$\Rightarrow Q2 = snd\: fork;rsh;[(add;D),(D^{-1};D)] $ \\[0.25cm]
$\Rightarrow Q2 = snd\: fork;rsh;[(add;D),id] = snd\: fork;rsh;fst(add;D)$
\\[0.5cm]
The new circuit can be seen below:
\begin{figure}[H]
\begin{center}
% Graphic for TeX using PGF
% Title: /home/yiannis/Desktop/cw-custom/diafiles/p2.dia
% Creator: Dia v0.97.2
% CreationDate: Mon Mar  4 05:09:02 2013
% For: yiannis
% \usepackage{tikz}
% The following commands are not supported in PSTricks at present
% We define them conditionally, so when they are implemented,
% this pgf file will use them.
\ifx\du\undefined
  \newlength{\du}
\fi
\setlength{\du}{15\unitlength}
\begin{tikzpicture}[thick, scale=0.6]
\pgftransformxscale{1.000000}
\pgftransformyscale{-1.000000}
\definecolor{dialinecolor}{rgb}{0.000000, 0.000000, 0.000000}
\pgfsetstrokecolor{dialinecolor}
\definecolor{dialinecolor}{rgb}{1.000000, 1.000000, 1.000000}
\pgfsetfillcolor{dialinecolor}
\definecolor{dialinecolor}{rgb}{1.000000, 1.000000, 1.000000}
\pgfsetfillcolor{dialinecolor}
\fill (-29.826100\du,-137.118879\du)--(-29.826100\du,-134.240223\du)--(-26.742093\du,-134.240223\du)--(-26.742093\du,-137.118879\du)--cycle;
\pgfsetlinewidth{0.100000\du}
\pgfsetdash{}{0pt}
\pgfsetdash{}{0pt}
\pgfsetmiterjoin
\definecolor{dialinecolor}{rgb}{0.000000, 0.000000, 0.000000}
\pgfsetstrokecolor{dialinecolor}
\draw (-29.826100\du,-137.118879\du)--(-29.826100\du,-134.240223\du)--(-26.742093\du,-134.240223\du)--(-26.742093\du,-137.118879\du)--cycle;
% setfont left to latex
\definecolor{dialinecolor}{rgb}{0.000000, 0.000000, 0.000000}
\pgfsetstrokecolor{dialinecolor}
\node at (-28.284097\du,-135.484551\du){ADD};
\pgfsetlinewidth{0.100000\du}
\pgfsetdash{}{0pt}
\pgfsetdash{}{0pt}
\pgfsetbuttcap
{
\definecolor{dialinecolor}{rgb}{0.000000, 0.000000, 0.000000}
\pgfsetfillcolor{dialinecolor}
% was here!!!
\definecolor{dialinecolor}{rgb}{0.000000, 0.000000, 0.000000}
\pgfsetstrokecolor{dialinecolor}
\draw (-26.742093\du,-135.679551\du)--(-25.026111\du,-135.676409\du);
}
\definecolor{dialinecolor}{rgb}{1.000000, 1.000000, 1.000000}
\pgfsetfillcolor{dialinecolor}
\fill (-25.026111\du,-137.115737\du)--(-25.026111\du,-134.237081\du)--(-21.942104\du,-134.237081\du)--(-21.942104\du,-137.115737\du)--cycle;
\pgfsetlinewidth{0.100000\du}
\pgfsetdash{}{0pt}
\pgfsetdash{}{0pt}
\pgfsetmiterjoin
\definecolor{dialinecolor}{rgb}{0.000000, 0.000000, 0.000000}
\pgfsetstrokecolor{dialinecolor}
\draw (-25.026111\du,-137.115737\du)--(-25.026111\du,-134.237081\du)--(-21.942104\du,-134.237081\du)--(-21.942104\du,-137.115737\du)--cycle;
% setfont left to latex
\definecolor{dialinecolor}{rgb}{0.000000, 0.000000, 0.000000}
\pgfsetstrokecolor{dialinecolor}
\node at (-23.484107\du,-135.481409\du){D};
\pgfsetlinewidth{0.100000\du}
\pgfsetdash{}{0pt}
\pgfsetdash{}{0pt}
\pgfsetbuttcap
{
\definecolor{dialinecolor}{rgb}{0.000000, 0.000000, 0.000000}
\pgfsetfillcolor{dialinecolor}
% was here!!!
\definecolor{dialinecolor}{rgb}{0.000000, 0.000000, 0.000000}
\pgfsetstrokecolor{dialinecolor}
\draw (-29.826100\du,-134.959887\du)--(-32.870160\du,-134.962888\du);
}
\pgfsetlinewidth{0.100000\du}
\pgfsetdash{}{0pt}
\pgfsetdash{}{0pt}
\pgfsetmiterjoin
\pgfsetbuttcap
{
\definecolor{dialinecolor}{rgb}{0.000000, 0.000000, 0.000000}
\pgfsetfillcolor{dialinecolor}
% was here!!!
{\pgfsetcornersarced{\pgfpoint{0.000000\du}{0.000000\du}}\definecolor{dialinecolor}{rgb}{0.000000, 0.000000, 0.000000}
\pgfsetstrokecolor{dialinecolor}
\draw (-29.826100\du,-136.399215\du)--(-30.876100\du,-136.399215\du)--(-30.876100\du,-138.049886\du)--(-19.905126\du,-138.049886\du);
}}
\definecolor{dialinecolor}{rgb}{1.000000, 1.000000, 1.000000}
\pgfsetfillcolor{dialinecolor}
\fill (-18.921344\du,-137.114675\du)--(-18.921344\du,-134.236019\du)--(-15.837337\du,-134.236019\du)--(-15.837337\du,-137.114675\du)--cycle;
\pgfsetlinewidth{0.100000\du}
\pgfsetdash{}{0pt}
\pgfsetdash{}{0pt}
\pgfsetmiterjoin
\definecolor{dialinecolor}{rgb}{0.000000, 0.000000, 0.000000}
\pgfsetstrokecolor{dialinecolor}
\draw (-18.921344\du,-137.114675\du)--(-18.921344\du,-134.236019\du)--(-15.837337\du,-134.236019\du)--(-15.837337\du,-137.114675\du)--cycle;
% setfont left to latex
\definecolor{dialinecolor}{rgb}{0.000000, 0.000000, 0.000000}
\pgfsetstrokecolor{dialinecolor}
\node at (-17.379340\du,-135.480347\du){ADD};
\pgfsetlinewidth{0.100000\du}
\pgfsetdash{}{0pt}
\pgfsetdash{}{0pt}
\pgfsetbuttcap
{
\definecolor{dialinecolor}{rgb}{0.000000, 0.000000, 0.000000}
\pgfsetfillcolor{dialinecolor}
% was here!!!
\definecolor{dialinecolor}{rgb}{0.000000, 0.000000, 0.000000}
\pgfsetstrokecolor{dialinecolor}
\draw (-15.837337\du,-135.675347\du)--(-14.121355\du,-135.672205\du);
}
\definecolor{dialinecolor}{rgb}{1.000000, 1.000000, 1.000000}
\pgfsetfillcolor{dialinecolor}
\fill (-14.121355\du,-137.111533\du)--(-14.121355\du,-134.232877\du)--(-11.037348\du,-134.232877\du)--(-11.037348\du,-137.111533\du)--cycle;
\pgfsetlinewidth{0.100000\du}
\pgfsetdash{}{0pt}
\pgfsetdash{}{0pt}
\pgfsetmiterjoin
\definecolor{dialinecolor}{rgb}{0.000000, 0.000000, 0.000000}
\pgfsetstrokecolor{dialinecolor}
\draw (-14.121355\du,-137.111533\du)--(-14.121355\du,-134.232877\du)--(-11.037348\du,-134.232877\du)--(-11.037348\du,-137.111533\du)--cycle;
% setfont left to latex
\definecolor{dialinecolor}{rgb}{0.000000, 0.000000, 0.000000}
\pgfsetstrokecolor{dialinecolor}
\node at (-12.579351\du,-135.477205\du){D};
\pgfsetlinewidth{0.100000\du}
\pgfsetdash{}{0pt}
\pgfsetdash{}{0pt}
\pgfsetbuttcap
{
\definecolor{dialinecolor}{rgb}{0.000000, 0.000000, 0.000000}
\pgfsetfillcolor{dialinecolor}
% was here!!!
\definecolor{dialinecolor}{rgb}{0.000000, 0.000000, 0.000000}
\pgfsetstrokecolor{dialinecolor}
\draw (-18.921344\du,-134.955683\du)--(-21.942104\du,-134.956745\du);
}
\pgfsetlinewidth{0.100000\du}
\pgfsetdash{}{0pt}
\pgfsetdash{}{0pt}
\pgfsetmiterjoin
\pgfsetbuttcap
{
\definecolor{dialinecolor}{rgb}{0.000000, 0.000000, 0.000000}
\pgfsetfillcolor{dialinecolor}
% was here!!!
{\pgfsetcornersarced{\pgfpoint{0.000000\du}{0.000000\du}}\definecolor{dialinecolor}{rgb}{0.000000, 0.000000, 0.000000}
\pgfsetstrokecolor{dialinecolor}
\draw (-18.921344\du,-136.395011\du)--(-19.971344\du,-136.395011\du)--(-19.971344\du,-138.046232\du)--(-9.013916\du,-138.046232\du);
}}
\definecolor{dialinecolor}{rgb}{1.000000, 1.000000, 1.000000}
\pgfsetfillcolor{dialinecolor}
\fill (-7.985275\du,-137.106538\du)--(-7.985275\du,-134.227882\du)--(-4.901268\du,-134.227882\du)--(-4.901268\du,-137.106538\du)--cycle;
\pgfsetlinewidth{0.100000\du}
\pgfsetdash{}{0pt}
\pgfsetdash{}{0pt}
\pgfsetmiterjoin
\definecolor{dialinecolor}{rgb}{0.000000, 0.000000, 0.000000}
\pgfsetstrokecolor{dialinecolor}
\draw (-7.985275\du,-137.106538\du)--(-7.985275\du,-134.227882\du)--(-4.901268\du,-134.227882\du)--(-4.901268\du,-137.106538\du)--cycle;
% setfont left to latex
\definecolor{dialinecolor}{rgb}{0.000000, 0.000000, 0.000000}
\pgfsetstrokecolor{dialinecolor}
\node at (-6.443272\du,-135.472210\du){ADD};
\pgfsetlinewidth{0.100000\du}
\pgfsetdash{}{0pt}
\pgfsetdash{}{0pt}
\pgfsetbuttcap
{
\definecolor{dialinecolor}{rgb}{0.000000, 0.000000, 0.000000}
\pgfsetfillcolor{dialinecolor}
% was here!!!
\definecolor{dialinecolor}{rgb}{0.000000, 0.000000, 0.000000}
\pgfsetstrokecolor{dialinecolor}
\draw (-4.901268\du,-135.667210\du)--(-3.185286\du,-135.664068\du);
}
\definecolor{dialinecolor}{rgb}{1.000000, 1.000000, 1.000000}
\pgfsetfillcolor{dialinecolor}
\fill (-3.185286\du,-137.103396\du)--(-3.185286\du,-134.224740\du)--(-0.101279\du,-134.224740\du)--(-0.101279\du,-137.103396\du)--cycle;
\pgfsetlinewidth{0.100000\du}
\pgfsetdash{}{0pt}
\pgfsetdash{}{0pt}
\pgfsetmiterjoin
\definecolor{dialinecolor}{rgb}{0.000000, 0.000000, 0.000000}
\pgfsetstrokecolor{dialinecolor}
\draw (-3.185286\du,-137.103396\du)--(-3.185286\du,-134.224740\du)--(-0.101279\du,-134.224740\du)--(-0.101279\du,-137.103396\du)--cycle;
% setfont left to latex
\definecolor{dialinecolor}{rgb}{0.000000, 0.000000, 0.000000}
\pgfsetstrokecolor{dialinecolor}
\node at (-1.643283\du,-135.469068\du){D};
\pgfsetlinewidth{0.100000\du}
\pgfsetdash{}{0pt}
\pgfsetdash{}{0pt}
\pgfsetbuttcap
{
\definecolor{dialinecolor}{rgb}{0.000000, 0.000000, 0.000000}
\pgfsetfillcolor{dialinecolor}
% was here!!!
\definecolor{dialinecolor}{rgb}{0.000000, 0.000000, 0.000000}
\pgfsetstrokecolor{dialinecolor}
\draw (-7.985275\du,-134.947546\du)--(-11.037348\du,-134.952541\du);
}
\pgfsetlinewidth{0.100000\du}
\pgfsetdash{}{0pt}
\pgfsetdash{}{0pt}
\pgfsetmiterjoin
\pgfsetbuttcap
{
\definecolor{dialinecolor}{rgb}{0.000000, 0.000000, 0.000000}
\pgfsetfillcolor{dialinecolor}
% was here!!!
{\pgfsetcornersarced{\pgfpoint{0.000000\du}{0.000000\du}}\definecolor{dialinecolor}{rgb}{0.000000, 0.000000, 0.000000}
\pgfsetstrokecolor{dialinecolor}
\draw (-7.985275\du,-136.386874\du)--(-9.035275\du,-136.386874\du)--(-9.035275\du,-138.044114\du)--(1.860543\du,-138.044114\du);
}}
\definecolor{dialinecolor}{rgb}{1.000000, 1.000000, 1.000000}
\pgfsetfillcolor{dialinecolor}
\fill (2.930504\du,-137.102778\du)--(2.930504\du,-134.224122\du)--(6.014511\du,-134.224122\du)--(6.014511\du,-137.102778\du)--cycle;
\pgfsetlinewidth{0.100000\du}
\pgfsetdash{}{0pt}
\pgfsetdash{}{0pt}
\pgfsetmiterjoin
\definecolor{dialinecolor}{rgb}{0.000000, 0.000000, 0.000000}
\pgfsetstrokecolor{dialinecolor}
\draw (2.930504\du,-137.102778\du)--(2.930504\du,-134.224122\du)--(6.014511\du,-134.224122\du)--(6.014511\du,-137.102778\du)--cycle;
% setfont left to latex
\definecolor{dialinecolor}{rgb}{0.000000, 0.000000, 0.000000}
\pgfsetstrokecolor{dialinecolor}
\node at (4.472507\du,-135.468450\du){ADD};
\pgfsetlinewidth{0.100000\du}
\pgfsetdash{}{0pt}
\pgfsetdash{}{0pt}
\pgfsetbuttcap
{
\definecolor{dialinecolor}{rgb}{0.000000, 0.000000, 0.000000}
\pgfsetfillcolor{dialinecolor}
% was here!!!
\definecolor{dialinecolor}{rgb}{0.000000, 0.000000, 0.000000}
\pgfsetstrokecolor{dialinecolor}
\draw (6.014511\du,-135.663450\du)--(7.730493\du,-135.660308\du);
}
\definecolor{dialinecolor}{rgb}{1.000000, 1.000000, 1.000000}
\pgfsetfillcolor{dialinecolor}
\fill (7.730493\du,-137.099636\du)--(7.730493\du,-134.220980\du)--(10.814500\du,-134.220980\du)--(10.814500\du,-137.099636\du)--cycle;
\pgfsetlinewidth{0.100000\du}
\pgfsetdash{}{0pt}
\pgfsetdash{}{0pt}
\pgfsetmiterjoin
\definecolor{dialinecolor}{rgb}{0.000000, 0.000000, 0.000000}
\pgfsetstrokecolor{dialinecolor}
\draw (7.730493\du,-137.099636\du)--(7.730493\du,-134.220980\du)--(10.814500\du,-134.220980\du)--(10.814500\du,-137.099636\du)--cycle;
% setfont left to latex
\definecolor{dialinecolor}{rgb}{0.000000, 0.000000, 0.000000}
\pgfsetstrokecolor{dialinecolor}
\node at (9.272496\du,-135.465308\du){D};
\pgfsetlinewidth{0.100000\du}
\pgfsetdash{}{0pt}
\pgfsetdash{}{0pt}
\pgfsetbuttcap
{
\definecolor{dialinecolor}{rgb}{0.000000, 0.000000, 0.000000}
\pgfsetfillcolor{dialinecolor}
% was here!!!
\definecolor{dialinecolor}{rgb}{0.000000, 0.000000, 0.000000}
\pgfsetstrokecolor{dialinecolor}
\draw (2.930504\du,-134.943786\du)--(-0.101279\du,-134.944404\du);
}
\pgfsetlinewidth{0.100000\du}
\pgfsetdash{}{0pt}
\pgfsetdash{}{0pt}
\pgfsetmiterjoin
\pgfsetbuttcap
{
\definecolor{dialinecolor}{rgb}{0.000000, 0.000000, 0.000000}
\pgfsetfillcolor{dialinecolor}
% was here!!!
{\pgfsetcornersarced{\pgfpoint{0.000000\du}{0.000000\du}}\definecolor{dialinecolor}{rgb}{0.000000, 0.000000, 0.000000}
\pgfsetstrokecolor{dialinecolor}
\draw (2.930504\du,-136.383114\du)--(1.880504\du,-136.383114\du)--(1.880504\du,-138.042639\du)--(10.821357\du,-138.042639\du);
}}
\pgfsetlinewidth{0.100000\du}
\pgfsetdash{}{0pt}
\pgfsetdash{}{0pt}
\pgfsetbuttcap
{
\definecolor{dialinecolor}{rgb}{0.000000, 0.000000, 0.000000}
\pgfsetfillcolor{dialinecolor}
% was here!!!
\definecolor{dialinecolor}{rgb}{0.000000, 0.000000, 0.000000}
\pgfsetstrokecolor{dialinecolor}
\draw (-30.824427\du,-138.047412\du)--(-32.994376\du,-138.062275\du);
}
\pgfsetlinewidth{0.100000\du}
\pgfsetdash{}{0pt}
\pgfsetdash{}{0pt}
\pgfsetbuttcap
{
\definecolor{dialinecolor}{rgb}{0.000000, 0.000000, 0.000000}
\pgfsetfillcolor{dialinecolor}
% was here!!!
\definecolor{dialinecolor}{rgb}{0.000000, 0.000000, 0.000000}
\pgfsetstrokecolor{dialinecolor}
\pgfpathmoveto{\pgfpoint{10.869077\du}{-134.905767\du}}
\pgfpatharc{94}{-97}{1.571667\du and 1.571667\du}
\pgfusepath{stroke}
}
\pgfsetlinewidth{0.100000\du}
\pgfsetdash{}{0pt}
\pgfsetdash{}{0pt}
\pgfsetbuttcap
{
\definecolor{dialinecolor}{rgb}{0.000000, 0.000000, 0.000000}
\pgfsetfillcolor{dialinecolor}
% was here!!!
\definecolor{dialinecolor}{rgb}{0.000000, 0.000000, 0.000000}
\pgfsetstrokecolor{dialinecolor}
\draw (12.553131\du,-136.555504\du)--(14.252849\du,-136.580500\du);
}
\end{tikzpicture}

\caption{The circuit after using the transformation to add the registers between the adders}
\end{center}
\end{figure}
The source code for the circuit is the following:
\begin{Verbatim}
INCLUDE "prelude.rby".
Q2 = snd fork;rsh;fst (add; D).
P2 n = Q2^n; fork^~1. 
current = P2 4.
\end{Verbatim}
The result can be seen below (for uninitialized delay)
\begin{Verbatim}
re "a;b;c"
Simulation start :

    0 - <a,?> ~ ?
    1 - <b,(? + ?)> ~ (? + ?)
    2 - <c,((? + ?) + (? + ?))> ~ ((? + ?) + (? + ?))

Simulation end :
\end{Verbatim}

\subsection*{(c) Slowdown}
Slowdown means doubling the registers of the design which trivially translates into combining the designs of Q1 and Q2 into a new design Q3 which has the registers of both.
This can be done by extending Q2 from $snd fork;rsh;fst (add; D)$ or $snd fork;rsh;[(add; D),id]$ to $snd fork;rsh;[(add; D),D^-1]$.\\[0.5cm]
The new circuit is the following:
\begin{figure}[H]
\begin{center}
% Graphic for TeX using PGF
% Title: /home/yiannis/Desktop/cw-custom/diafiles/p3.dia
% Creator: Dia v0.97.2
% CreationDate: Mon Mar  4 20:21:16 2013
% For: yiannis
% \usepackage{tikz}
% The following commands are not supported in PSTricks at present
% We define them conditionally, so when they are implemented,
% this pgf file will use them.
\ifx\du\undefined
  \newlength{\du}
\fi
\setlength{\du}{15\unitlength}
\begin{tikzpicture}[thick, scale=0.6]
\pgftransformxscale{1.000000}
\pgftransformyscale{-1.000000}
\definecolor{dialinecolor}{rgb}{0.000000, 0.000000, 0.000000}
\pgfsetstrokecolor{dialinecolor}
\definecolor{dialinecolor}{rgb}{1.000000, 1.000000, 1.000000}
\pgfsetfillcolor{dialinecolor}
\definecolor{dialinecolor}{rgb}{1.000000, 1.000000, 1.000000}
\pgfsetfillcolor{dialinecolor}
\fill (-29.826100\du,-137.119000\du)--(-29.826100\du,-134.240344\du)--(-26.742093\du,-134.240344\du)--(-26.742093\du,-137.119000\du)--cycle;
\pgfsetlinewidth{0.100000\du}
\pgfsetdash{}{0pt}
\pgfsetdash{}{0pt}
\pgfsetmiterjoin
\definecolor{dialinecolor}{rgb}{0.000000, 0.000000, 0.000000}
\pgfsetstrokecolor{dialinecolor}
\draw (-29.826100\du,-137.119000\du)--(-29.826100\du,-134.240344\du)--(-26.742093\du,-134.240344\du)--(-26.742093\du,-137.119000\du)--cycle;
% setfont left to latex
\definecolor{dialinecolor}{rgb}{0.000000, 0.000000, 0.000000}
\pgfsetstrokecolor{dialinecolor}
\node at (-28.284096\du,-135.484672\du){ADD};
\pgfsetlinewidth{0.100000\du}
\pgfsetdash{}{0pt}
\pgfsetdash{}{0pt}
\pgfsetbuttcap
{
\definecolor{dialinecolor}{rgb}{0.000000, 0.000000, 0.000000}
\pgfsetfillcolor{dialinecolor}
% was here!!!
\definecolor{dialinecolor}{rgb}{0.000000, 0.000000, 0.000000}
\pgfsetstrokecolor{dialinecolor}
\draw (-26.742100\du,-135.680000\du)--(-25.026100\du,-135.676000\du);
}
\definecolor{dialinecolor}{rgb}{1.000000, 1.000000, 1.000000}
\pgfsetfillcolor{dialinecolor}
\fill (-25.026100\du,-137.116000\du)--(-25.026100\du,-134.237344\du)--(-21.942093\du,-134.237344\du)--(-21.942093\du,-137.116000\du)--cycle;
\pgfsetlinewidth{0.100000\du}
\pgfsetdash{}{0pt}
\pgfsetdash{}{0pt}
\pgfsetmiterjoin
\definecolor{dialinecolor}{rgb}{0.000000, 0.000000, 0.000000}
\pgfsetstrokecolor{dialinecolor}
\draw (-25.026100\du,-137.116000\du)--(-25.026100\du,-134.237344\du)--(-21.942093\du,-134.237344\du)--(-21.942093\du,-137.116000\du)--cycle;
% setfont left to latex
\definecolor{dialinecolor}{rgb}{0.000000, 0.000000, 0.000000}
\pgfsetstrokecolor{dialinecolor}
\node at (-23.484096\du,-135.481672\du){D};
\pgfsetlinewidth{0.100000\du}
\pgfsetdash{}{0pt}
\pgfsetdash{}{0pt}
\pgfsetbuttcap
{
\definecolor{dialinecolor}{rgb}{0.000000, 0.000000, 0.000000}
\pgfsetfillcolor{dialinecolor}
% was here!!!
\definecolor{dialinecolor}{rgb}{0.000000, 0.000000, 0.000000}
\pgfsetstrokecolor{dialinecolor}
\draw (-29.826100\du,-134.960000\du)--(-32.870200\du,-134.963000\du);
}
\pgfsetlinewidth{0.100000\du}
\pgfsetdash{}{0pt}
\pgfsetdash{}{0pt}
\pgfsetmiterjoin
\pgfsetbuttcap
{
\definecolor{dialinecolor}{rgb}{0.000000, 0.000000, 0.000000}
\pgfsetfillcolor{dialinecolor}
% was here!!!
{\pgfsetcornersarced{\pgfpoint{0.000000\du}{0.000000\du}}\definecolor{dialinecolor}{rgb}{0.000000, 0.000000, 0.000000}
\pgfsetstrokecolor{dialinecolor}
\draw (-29.826100\du,-136.399336\du)--(-30.876100\du,-136.399336\du)--(-30.876100\du,-138.050000\du)--(-19.905100\du,-138.050000\du);
}}
\definecolor{dialinecolor}{rgb}{1.000000, 1.000000, 1.000000}
\pgfsetfillcolor{dialinecolor}
\fill (-18.921300\du,-137.115000\du)--(-18.921300\du,-134.236344\du)--(-15.837293\du,-134.236344\du)--(-15.837293\du,-137.115000\du)--cycle;
\pgfsetlinewidth{0.100000\du}
\pgfsetdash{}{0pt}
\pgfsetdash{}{0pt}
\pgfsetmiterjoin
\definecolor{dialinecolor}{rgb}{0.000000, 0.000000, 0.000000}
\pgfsetstrokecolor{dialinecolor}
\draw (-18.921300\du,-137.115000\du)--(-18.921300\du,-134.236344\du)--(-15.837293\du,-134.236344\du)--(-15.837293\du,-137.115000\du)--cycle;
% setfont left to latex
\definecolor{dialinecolor}{rgb}{0.000000, 0.000000, 0.000000}
\pgfsetstrokecolor{dialinecolor}
\node at (-17.379296\du,-135.480672\du){ADD};
\pgfsetlinewidth{0.100000\du}
\pgfsetdash{}{0pt}
\pgfsetdash{}{0pt}
\pgfsetbuttcap
{
\definecolor{dialinecolor}{rgb}{0.000000, 0.000000, 0.000000}
\pgfsetfillcolor{dialinecolor}
% was here!!!
\definecolor{dialinecolor}{rgb}{0.000000, 0.000000, 0.000000}
\pgfsetstrokecolor{dialinecolor}
\draw (-15.837300\du,-135.675000\du)--(-14.121400\du,-135.672000\du);
}
\definecolor{dialinecolor}{rgb}{1.000000, 1.000000, 1.000000}
\pgfsetfillcolor{dialinecolor}
\fill (-14.121400\du,-137.112000\du)--(-14.121400\du,-134.233344\du)--(-11.037393\du,-134.233344\du)--(-11.037393\du,-137.112000\du)--cycle;
\pgfsetlinewidth{0.100000\du}
\pgfsetdash{}{0pt}
\pgfsetdash{}{0pt}
\pgfsetmiterjoin
\definecolor{dialinecolor}{rgb}{0.000000, 0.000000, 0.000000}
\pgfsetstrokecolor{dialinecolor}
\draw (-14.121400\du,-137.112000\du)--(-14.121400\du,-134.233344\du)--(-11.037393\du,-134.233344\du)--(-11.037393\du,-137.112000\du)--cycle;
% setfont left to latex
\definecolor{dialinecolor}{rgb}{0.000000, 0.000000, 0.000000}
\pgfsetstrokecolor{dialinecolor}
\node at (-12.579396\du,-135.477672\du){D};
\pgfsetlinewidth{0.100000\du}
\pgfsetdash{}{0pt}
\pgfsetdash{}{0pt}
\pgfsetbuttcap
{
\definecolor{dialinecolor}{rgb}{0.000000, 0.000000, 0.000000}
\pgfsetfillcolor{dialinecolor}
% was here!!!
\definecolor{dialinecolor}{rgb}{0.000000, 0.000000, 0.000000}
\pgfsetstrokecolor{dialinecolor}
\draw (-18.921300\du,-134.956000\du)--(-21.942100\du,-134.957000\du);
}
\pgfsetlinewidth{0.100000\du}
\pgfsetdash{}{0pt}
\pgfsetdash{}{0pt}
\pgfsetmiterjoin
\pgfsetbuttcap
{
\definecolor{dialinecolor}{rgb}{0.000000, 0.000000, 0.000000}
\pgfsetfillcolor{dialinecolor}
% was here!!!
{\pgfsetcornersarced{\pgfpoint{0.000000\du}{0.000000\du}}\definecolor{dialinecolor}{rgb}{0.000000, 0.000000, 0.000000}
\pgfsetstrokecolor{dialinecolor}
\draw (-18.921300\du,-136.395336\du)--(-19.971300\du,-136.395336\du)--(-19.971300\du,-138.046000\du)--(-9.013920\du,-138.046000\du);
}}
\definecolor{dialinecolor}{rgb}{1.000000, 1.000000, 1.000000}
\pgfsetfillcolor{dialinecolor}
\fill (-7.985280\du,-137.107000\du)--(-7.985280\du,-134.228344\du)--(-4.901273\du,-134.228344\du)--(-4.901273\du,-137.107000\du)--cycle;
\pgfsetlinewidth{0.100000\du}
\pgfsetdash{}{0pt}
\pgfsetdash{}{0pt}
\pgfsetmiterjoin
\definecolor{dialinecolor}{rgb}{0.000000, 0.000000, 0.000000}
\pgfsetstrokecolor{dialinecolor}
\draw (-7.985280\du,-137.107000\du)--(-7.985280\du,-134.228344\du)--(-4.901273\du,-134.228344\du)--(-4.901273\du,-137.107000\du)--cycle;
% setfont left to latex
\definecolor{dialinecolor}{rgb}{0.000000, 0.000000, 0.000000}
\pgfsetstrokecolor{dialinecolor}
\node at (-6.443276\du,-135.472672\du){ADD};
\pgfsetlinewidth{0.100000\du}
\pgfsetdash{}{0pt}
\pgfsetdash{}{0pt}
\pgfsetbuttcap
{
\definecolor{dialinecolor}{rgb}{0.000000, 0.000000, 0.000000}
\pgfsetfillcolor{dialinecolor}
% was here!!!
\definecolor{dialinecolor}{rgb}{0.000000, 0.000000, 0.000000}
\pgfsetstrokecolor{dialinecolor}
\draw (-4.901270\du,-135.667000\du)--(-3.185290\du,-135.664000\du);
}
\definecolor{dialinecolor}{rgb}{1.000000, 1.000000, 1.000000}
\pgfsetfillcolor{dialinecolor}
\fill (-3.185290\du,-137.103000\du)--(-3.185290\du,-134.224344\du)--(-0.101283\du,-134.224344\du)--(-0.101283\du,-137.103000\du)--cycle;
\pgfsetlinewidth{0.100000\du}
\pgfsetdash{}{0pt}
\pgfsetdash{}{0pt}
\pgfsetmiterjoin
\definecolor{dialinecolor}{rgb}{0.000000, 0.000000, 0.000000}
\pgfsetstrokecolor{dialinecolor}
\draw (-3.185290\du,-137.103000\du)--(-3.185290\du,-134.224344\du)--(-0.101283\du,-134.224344\du)--(-0.101283\du,-137.103000\du)--cycle;
% setfont left to latex
\definecolor{dialinecolor}{rgb}{0.000000, 0.000000, 0.000000}
\pgfsetstrokecolor{dialinecolor}
\node at (-1.643286\du,-135.468672\du){D};
\pgfsetlinewidth{0.100000\du}
\pgfsetdash{}{0pt}
\pgfsetdash{}{0pt}
\pgfsetbuttcap
{
\definecolor{dialinecolor}{rgb}{0.000000, 0.000000, 0.000000}
\pgfsetfillcolor{dialinecolor}
% was here!!!
\definecolor{dialinecolor}{rgb}{0.000000, 0.000000, 0.000000}
\pgfsetstrokecolor{dialinecolor}
\draw (-7.985280\du,-134.948000\du)--(-11.037300\du,-134.953000\du);
}
\pgfsetlinewidth{0.100000\du}
\pgfsetdash{}{0pt}
\pgfsetdash{}{0pt}
\pgfsetmiterjoin
\pgfsetbuttcap
{
\definecolor{dialinecolor}{rgb}{0.000000, 0.000000, 0.000000}
\pgfsetfillcolor{dialinecolor}
% was here!!!
{\pgfsetcornersarced{\pgfpoint{0.000000\du}{0.000000\du}}\definecolor{dialinecolor}{rgb}{0.000000, 0.000000, 0.000000}
\pgfsetstrokecolor{dialinecolor}
\draw (-7.985280\du,-136.387336\du)--(-9.035280\du,-136.387336\du)--(-9.035280\du,-138.044000\du)--(1.860540\du,-138.044000\du);
}}
\definecolor{dialinecolor}{rgb}{1.000000, 1.000000, 1.000000}
\pgfsetfillcolor{dialinecolor}
\fill (2.930500\du,-137.103000\du)--(2.930500\du,-134.224344\du)--(6.014507\du,-134.224344\du)--(6.014507\du,-137.103000\du)--cycle;
\pgfsetlinewidth{0.100000\du}
\pgfsetdash{}{0pt}
\pgfsetdash{}{0pt}
\pgfsetmiterjoin
\definecolor{dialinecolor}{rgb}{0.000000, 0.000000, 0.000000}
\pgfsetstrokecolor{dialinecolor}
\draw (2.930500\du,-137.103000\du)--(2.930500\du,-134.224344\du)--(6.014507\du,-134.224344\du)--(6.014507\du,-137.103000\du)--cycle;
% setfont left to latex
\definecolor{dialinecolor}{rgb}{0.000000, 0.000000, 0.000000}
\pgfsetstrokecolor{dialinecolor}
\node at (4.472504\du,-135.468672\du){ADD};
\pgfsetlinewidth{0.100000\du}
\pgfsetdash{}{0pt}
\pgfsetdash{}{0pt}
\pgfsetbuttcap
{
\definecolor{dialinecolor}{rgb}{0.000000, 0.000000, 0.000000}
\pgfsetfillcolor{dialinecolor}
% was here!!!
\definecolor{dialinecolor}{rgb}{0.000000, 0.000000, 0.000000}
\pgfsetstrokecolor{dialinecolor}
\draw (6.014510\du,-135.663000\du)--(7.730490\du,-135.660000\du);
}
\definecolor{dialinecolor}{rgb}{1.000000, 1.000000, 1.000000}
\pgfsetfillcolor{dialinecolor}
\fill (7.730490\du,-137.100000\du)--(7.730490\du,-134.221344\du)--(10.814497\du,-134.221344\du)--(10.814497\du,-137.100000\du)--cycle;
\pgfsetlinewidth{0.100000\du}
\pgfsetdash{}{0pt}
\pgfsetdash{}{0pt}
\pgfsetmiterjoin
\definecolor{dialinecolor}{rgb}{0.000000, 0.000000, 0.000000}
\pgfsetstrokecolor{dialinecolor}
\draw (7.730490\du,-137.100000\du)--(7.730490\du,-134.221344\du)--(10.814497\du,-134.221344\du)--(10.814497\du,-137.100000\du)--cycle;
% setfont left to latex
\definecolor{dialinecolor}{rgb}{0.000000, 0.000000, 0.000000}
\pgfsetstrokecolor{dialinecolor}
\node at (9.272494\du,-135.465672\du){D};
\pgfsetlinewidth{0.100000\du}
\pgfsetdash{}{0pt}
\pgfsetdash{}{0pt}
\pgfsetbuttcap
{
\definecolor{dialinecolor}{rgb}{0.000000, 0.000000, 0.000000}
\pgfsetfillcolor{dialinecolor}
% was here!!!
\definecolor{dialinecolor}{rgb}{0.000000, 0.000000, 0.000000}
\pgfsetstrokecolor{dialinecolor}
\draw (2.930500\du,-134.944000\du)--(-0.101279\du,-134.944000\du);
}
\pgfsetlinewidth{0.100000\du}
\pgfsetdash{}{0pt}
\pgfsetdash{}{0pt}
\pgfsetmiterjoin
\pgfsetbuttcap
{
\definecolor{dialinecolor}{rgb}{0.000000, 0.000000, 0.000000}
\pgfsetfillcolor{dialinecolor}
% was here!!!
{\pgfsetcornersarced{\pgfpoint{0.000000\du}{0.000000\du}}\definecolor{dialinecolor}{rgb}{0.000000, 0.000000, 0.000000}
\pgfsetstrokecolor{dialinecolor}
\draw (2.930500\du,-136.383336\du)--(1.880500\du,-136.383336\du)--(1.880500\du,-138.043000\du)--(10.821400\du,-138.043000\du);
}}
\pgfsetlinewidth{0.100000\du}
\pgfsetdash{}{0pt}
\pgfsetdash{}{0pt}
\pgfsetbuttcap
{
\definecolor{dialinecolor}{rgb}{0.000000, 0.000000, 0.000000}
\pgfsetfillcolor{dialinecolor}
% was here!!!
\definecolor{dialinecolor}{rgb}{0.000000, 0.000000, 0.000000}
\pgfsetstrokecolor{dialinecolor}
\draw (-30.824400\du,-138.047000\du)--(-32.994400\du,-138.062000\du);
}
\pgfsetlinewidth{0.100000\du}
\pgfsetdash{}{0pt}
\pgfsetdash{}{0pt}
\pgfsetbuttcap
{
\definecolor{dialinecolor}{rgb}{0.000000, 0.000000, 0.000000}
\pgfsetfillcolor{dialinecolor}
% was here!!!
\definecolor{dialinecolor}{rgb}{0.000000, 0.000000, 0.000000}
\pgfsetstrokecolor{dialinecolor}
\pgfpathmoveto{\pgfpoint{10.869077\du}{-134.906006\du}}
\pgfpatharc{94}{-97}{1.571778\du and 1.571778\du}
\pgfusepath{stroke}
}
\pgfsetlinewidth{0.100000\du}
\pgfsetdash{}{0pt}
\pgfsetdash{}{0pt}
\pgfsetbuttcap
{
\definecolor{dialinecolor}{rgb}{0.000000, 0.000000, 0.000000}
\pgfsetfillcolor{dialinecolor}
% was here!!!
\definecolor{dialinecolor}{rgb}{0.000000, 0.000000, 0.000000}
\pgfsetstrokecolor{dialinecolor}
\draw (12.553100\du,-136.556000\du)--(14.252800\du,-136.580000\du);
}
\definecolor{dialinecolor}{rgb}{1.000000, 1.000000, 1.000000}
\pgfsetfillcolor{dialinecolor}
\fill (5.617699\du,-139.351716\du)--(5.617699\du,-137.451716\du)--(7.847699\du,-137.451716\du)--(7.847699\du,-139.351716\du)--cycle;
\pgfsetlinewidth{0.100000\du}
\pgfsetdash{}{0pt}
\pgfsetdash{}{0pt}
\pgfsetmiterjoin
\definecolor{dialinecolor}{rgb}{0.000000, 0.000000, 0.000000}
\pgfsetstrokecolor{dialinecolor}
\draw (5.617699\du,-139.351716\du)--(5.617699\du,-137.451716\du)--(7.847699\du,-137.451716\du)--(7.847699\du,-139.351716\du)--cycle;
% setfont left to latex
\definecolor{dialinecolor}{rgb}{0.000000, 0.000000, 0.000000}
\pgfsetstrokecolor{dialinecolor}
\node at (6.732699\du,-138.206716\du){D-1};
\definecolor{dialinecolor}{rgb}{1.000000, 1.000000, 1.000000}
\pgfsetfillcolor{dialinecolor}
\fill (-5.282693\du,-139.349862\du)--(-5.282693\du,-137.449862\du)--(-3.052693\du,-137.449862\du)--(-3.052693\du,-139.349862\du)--cycle;
\pgfsetlinewidth{0.100000\du}
\pgfsetdash{}{0pt}
\pgfsetdash{}{0pt}
\pgfsetmiterjoin
\definecolor{dialinecolor}{rgb}{0.000000, 0.000000, 0.000000}
\pgfsetstrokecolor{dialinecolor}
\draw (-5.282693\du,-139.349862\du)--(-5.282693\du,-137.449862\du)--(-3.052693\du,-137.449862\du)--(-3.052693\du,-139.349862\du)--cycle;
% setfont left to latex
\definecolor{dialinecolor}{rgb}{0.000000, 0.000000, 0.000000}
\pgfsetstrokecolor{dialinecolor}
\node at (-4.167693\du,-138.204862\du){D-1};
\definecolor{dialinecolor}{rgb}{1.000000, 1.000000, 1.000000}
\pgfsetfillcolor{dialinecolor}
\fill (-16.103265\du,-139.349862\du)--(-16.103265\du,-137.449862\du)--(-13.873265\du,-137.449862\du)--(-13.873265\du,-139.349862\du)--cycle;
\pgfsetlinewidth{0.100000\du}
\pgfsetdash{}{0pt}
\pgfsetdash{}{0pt}
\pgfsetmiterjoin
\definecolor{dialinecolor}{rgb}{0.000000, 0.000000, 0.000000}
\pgfsetstrokecolor{dialinecolor}
\draw (-16.103265\du,-139.349862\du)--(-16.103265\du,-137.449862\du)--(-13.873265\du,-137.449862\du)--(-13.873265\du,-139.349862\du)--cycle;
% setfont left to latex
\definecolor{dialinecolor}{rgb}{0.000000, 0.000000, 0.000000}
\pgfsetstrokecolor{dialinecolor}
\node at (-14.988265\du,-138.204862\du){D-1};
\definecolor{dialinecolor}{rgb}{1.000000, 1.000000, 1.000000}
\pgfsetfillcolor{dialinecolor}
\fill (-27.181389\du,-139.395712\du)--(-27.181389\du,-137.495712\du)--(-24.619685\du,-137.495712\du)--(-24.619685\du,-139.395712\du)--cycle;
\pgfsetlinewidth{0.100000\du}
\pgfsetdash{}{0pt}
\pgfsetdash{}{0pt}
\pgfsetmiterjoin
\definecolor{dialinecolor}{rgb}{0.000000, 0.000000, 0.000000}
\pgfsetstrokecolor{dialinecolor}
\draw (-27.181389\du,-139.395712\du)--(-27.181389\du,-137.495712\du)--(-24.619685\du,-137.495712\du)--(-24.619685\du,-139.395712\du)--cycle;
% setfont left to latex
\definecolor{dialinecolor}{rgb}{0.000000, 0.000000, 0.000000}
\pgfsetstrokecolor{dialinecolor}
\node at (-25.900537\du,-138.250712\du){D-1};
\end{tikzpicture}

\caption{The circuit after slowdown}
\end{center}
\end{figure}
The source code:
\begin{Verbatim}
INCLUDE "prelude.rby".
Q3 = snd fork;rsh;[(add; D),D^~1].
P3 n = Q3^n; fork^~1. 
current = P3 4.
\end{Verbatim}
The result:
\begin{Verbatim}
re "a;b;c"
Simulation start :

    0 - <a,?> ~ ?
    1 - <b,?> ~ (? + ?)
    2 - <c,?> ~ ((? + ?) + ?)

Simulation end :
\end{Verbatim}
\section*{Question 3}
\subsection*{(a) Proof by induction}
In order to show that $[P,Q]^{n};R = R;Q^n$ for $n>0$, we first have to show that it is $True$ for n = 1.
\\[0.5cm]
\textbf{Base case:} $[P,Q]^{1};R=R;Q^1$
\\[0.5cm]
This is intuitively shown to be true by the given assumption $[P,Q]^{n};R=R;Q$ which is equivalent.
\\[0.5cm]
Assuming that it is also true for  $n = k > 0$ 
\\[0.5cm]
\textbf{Inductive Hypothesis:} $[P,Q]^{k};R = R;Q^k$
\\[0.5cm]
We need to show that the same is true for $n = k+1$ and $[P,Q]^{k+1};R = R;Q^{k+1}$
\\[0.5cm]
 $[P,Q]^{k+1};R $ \textbf{LHS}
\\[0.25cm]
$= [P,Q]^k;[P,Q];R$ (by sequential expansion of $[P,Q]^{k+1}$)
\\[0.25cm]
$= [P,Q]^k;R;Q$ (since $[P,Q];R = R;Q$ given)
\\[0.25cm]
$= R;Q^k;Q$ (by the i.h. $[P,Q]^{k};R = R;Q^k$)
\\[0.25cm]
$= R;Q^{k+1}$ (by sequential contraction of $Q^{k};Q$)  \textbf{RHS}
\\[0.5cm]
So by induction we have \textbf{proved} that if $[P,Q];R = R;Q$ is given to be $True$, for $n>0$:
\\[0.5cm]
$[P,Q]^{n};R = R;Q^n$ 
is also $True$
\subsection*{(b) Inductive Definitions}
\subsubsection*{Right-reduction}
$rdr_1 = fst\:  [-]^{-1};R.$ \\[0.25cm]
$rdr_{n+1} = fst\:  apl_{n}^{-1};lsh;snd(rdr_n\: R);R.$
\subsubsection*{Delta (triangle)}
$\Delta_0 = [\: ].$ \\[0.25cm]
$\Delta_{n+1} = [\Delta_{n},R^n]\backslash apr_n.$
\subsection*{(c) Horner's Rule}
\subsubsection*{Left-hand side}
\begin{figure}[H]
\begin{center}
% Graphic for TeX using PGF
% Title: /home/yiannis/Desktop/cw-custom/diafiles/hor1.dia
% Creator: Dia v0.97.2
% CreationDate: Sat Mar  2 03:24:31 2013
% For: yiannis
% \usepackage{tikz}
% The following commands are not supported in PSTricks at present
% % We define them conditionally, so when they are implemented,
% this pgf file will use them.
\ifx\du\undefined
  \newlength{\du}
\fi
\setlength{\du}{15\unitlength}
\begin{tikzpicture}[thick, scale=0.4]
\pgftransformxscale{1.000000}
\pgftransformyscale{-1.000000}
\definecolor{dialinecolor}{rgb}{0.000000, 0.000000, 0.000000}
\pgfsetstrokecolor{dialinecolor}
\definecolor{dialinecolor}{rgb}{1.000000, 1.000000, 1.000000}
\pgfsetfillcolor{dialinecolor}
\definecolor{dialinecolor}{rgb}{1.000000, 1.000000, 1.000000}
\pgfsetfillcolor{dialinecolor}
\pgfpathellipse{\pgfpoint{33.846636\du}{10.898318\du}}{\pgfpoint{1.503364\du}{0\du}}{\pgfpoint{0\du}{1.501682\du}}
\pgfusepath{fill}
\pgfsetlinewidth{0.100000\du}
\pgfsetdash{}{0pt}
\pgfsetdash{}{0pt}
\pgfsetmiterjoin
\definecolor{dialinecolor}{rgb}{0.000000, 0.000000, 0.000000}
\pgfsetstrokecolor{dialinecolor}
\pgfpathellipse{\pgfpoint{33.846636\du}{10.898318\du}}{\pgfpoint{1.503364\du}{0\du}}{\pgfpoint{0\du}{1.501682\du}}
\pgfusepath{stroke}
% setfont left to latex
\definecolor{dialinecolor}{rgb}{0.000000, 0.000000, 0.000000}
\pgfsetstrokecolor{dialinecolor}
\node at (33.846636\du,11.093318\du){Q};
\definecolor{dialinecolor}{rgb}{1.000000, 1.000000, 1.000000}
\pgfsetfillcolor{dialinecolor}
\pgfpathellipse{\pgfpoint{33.843364\du}{15.361682\du}}{\pgfpoint{1.503364\du}{0\du}}{\pgfpoint{0\du}{1.501682\du}}
\pgfusepath{fill}
\pgfsetlinewidth{0.100000\du}
\pgfsetdash{}{0pt}
\pgfsetdash{}{0pt}
\pgfsetmiterjoin
\definecolor{dialinecolor}{rgb}{0.000000, 0.000000, 0.000000}
\pgfsetstrokecolor{dialinecolor}
\pgfpathellipse{\pgfpoint{33.843364\du}{15.361682\du}}{\pgfpoint{1.503364\du}{0\du}}{\pgfpoint{0\du}{1.501682\du}}
\pgfusepath{stroke}
% setfont left to latex
\definecolor{dialinecolor}{rgb}{0.000000, 0.000000, 0.000000}
\pgfsetstrokecolor{dialinecolor}
\node at (33.843364\du,15.556682\du){Q};
\pgfsetlinewidth{0.100000\du}
\pgfsetdash{}{0pt}
\pgfsetdash{}{0pt}
\pgfsetbuttcap
{
\definecolor{dialinecolor}{rgb}{0.000000, 0.000000, 0.000000}
\pgfsetfillcolor{dialinecolor}
% was here!!!
\definecolor{dialinecolor}{rgb}{0.000000, 0.000000, 0.000000}
\pgfsetstrokecolor{dialinecolor}
\draw (33.846636\du,12.400000\du)--(33.843364\du,13.860000\du);
}
\definecolor{dialinecolor}{rgb}{1.000000, 1.000000, 1.000000}
\pgfsetfillcolor{dialinecolor}
\pgfpathellipse{\pgfpoint{33.840524\du}{19.835142\du}}{\pgfpoint{1.503364\du}{0\du}}{\pgfpoint{0\du}{1.501682\du}}
\pgfusepath{fill}
\pgfsetlinewidth{0.100000\du}
\pgfsetdash{}{0pt}
\pgfsetdash{}{0pt}
\pgfsetmiterjoin
\definecolor{dialinecolor}{rgb}{0.000000, 0.000000, 0.000000}
\pgfsetstrokecolor{dialinecolor}
\pgfpathellipse{\pgfpoint{33.840524\du}{19.835142\du}}{\pgfpoint{1.503364\du}{0\du}}{\pgfpoint{0\du}{1.501682\du}}
\pgfusepath{stroke}
% setfont left to latex
\definecolor{dialinecolor}{rgb}{0.000000, 0.000000, 0.000000}
\pgfsetstrokecolor{dialinecolor}
\node at (33.840524\du,20.030142\du){Q};
\pgfsetlinewidth{0.100000\du}
\pgfsetdash{}{0pt}
\pgfsetdash{}{0pt}
\pgfsetbuttcap
{
\definecolor{dialinecolor}{rgb}{0.000000, 0.000000, 0.000000}
\pgfsetfillcolor{dialinecolor}
% was here!!!
\definecolor{dialinecolor}{rgb}{0.000000, 0.000000, 0.000000}
\pgfsetstrokecolor{dialinecolor}
\draw (33.843364\du,16.863364\du)--(33.840524\du,18.333460\du);
}
\pgfsetlinewidth{0.100000\du}
\pgfsetdash{}{0pt}
\pgfsetdash{}{0pt}
\pgfsetbuttcap
{
\definecolor{dialinecolor}{rgb}{0.000000, 0.000000, 0.000000}
\pgfsetfillcolor{dialinecolor}
% was here!!!
\definecolor{dialinecolor}{rgb}{0.000000, 0.000000, 0.000000}
\pgfsetstrokecolor{dialinecolor}
\draw (33.846636\du,9.396636\du)--(33.860208\du,7.442894\du);
}
\definecolor{dialinecolor}{rgb}{1.000000, 1.000000, 1.000000}
\pgfsetfillcolor{dialinecolor}
\fill (32.270596\du,22.670264\du)--(32.270596\du,25.855242\du)--(35.427249\du,25.855242\du)--(35.427249\du,22.670264\du)--cycle;
\pgfsetlinewidth{0.100000\du}
\pgfsetdash{}{0pt}
\pgfsetdash{}{0pt}
\pgfsetmiterjoin
\definecolor{dialinecolor}{rgb}{0.000000, 0.000000, 0.000000}
\pgfsetstrokecolor{dialinecolor}
\draw (32.270596\du,22.670264\du)--(32.270596\du,25.855242\du)--(35.427249\du,25.855242\du)--(35.427249\du,22.670264\du)--cycle;
% setfont left to latex
\definecolor{dialinecolor}{rgb}{0.000000, 0.000000, 0.000000}
\pgfsetstrokecolor{dialinecolor}
\node at (33.848922\du,24.457753\du){R};
\definecolor{dialinecolor}{rgb}{1.000000, 1.000000, 1.000000}
\pgfsetfillcolor{dialinecolor}
\fill (32.273289\du,27.509855\du)--(32.273289\du,30.694834\du)--(35.429942\du,30.694834\du)--(35.429942\du,27.509855\du)--cycle;
\pgfsetlinewidth{0.100000\du}
\pgfsetdash{}{0pt}
\pgfsetdash{}{0pt}
\pgfsetmiterjoin
\definecolor{dialinecolor}{rgb}{0.000000, 0.000000, 0.000000}
\pgfsetstrokecolor{dialinecolor}
\draw (32.273289\du,27.509855\du)--(32.273289\du,30.694834\du)--(35.429942\du,30.694834\du)--(35.429942\du,27.509855\du)--cycle;
% setfont left to latex
\definecolor{dialinecolor}{rgb}{0.000000, 0.000000, 0.000000}
\pgfsetstrokecolor{dialinecolor}
\node at (33.851616\du,29.297345\du){R};
\pgfsetlinewidth{0.100000\du}
\pgfsetdash{}{0pt}
\pgfsetdash{}{0pt}
\pgfsetbuttcap
{
\definecolor{dialinecolor}{rgb}{0.000000, 0.000000, 0.000000}
\pgfsetfillcolor{dialinecolor}
% was here!!!
\definecolor{dialinecolor}{rgb}{0.000000, 0.000000, 0.000000}
\pgfsetstrokecolor{dialinecolor}
\draw (33.848922\du,25.855242\du)--(33.851616\du,27.509855\du);
}
\definecolor{dialinecolor}{rgb}{1.000000, 1.000000, 1.000000}
\pgfsetfillcolor{dialinecolor}
\fill (32.257805\du,32.366566\du)--(32.257805\du,35.551544\du)--(35.414457\du,35.551544\du)--(35.414457\du,32.366566\du)--cycle;
\pgfsetlinewidth{0.100000\du}
\pgfsetdash{}{0pt}
\pgfsetdash{}{0pt}
\pgfsetmiterjoin
\definecolor{dialinecolor}{rgb}{0.000000, 0.000000, 0.000000}
\pgfsetstrokecolor{dialinecolor}
\draw (32.257805\du,32.366566\du)--(32.257805\du,35.551544\du)--(35.414457\du,35.551544\du)--(35.414457\du,32.366566\du)--cycle;
% setfont left to latex
\definecolor{dialinecolor}{rgb}{0.000000, 0.000000, 0.000000}
\pgfsetstrokecolor{dialinecolor}
\node at (33.836131\du,34.154055\du){R};
\pgfsetlinewidth{0.100000\du}
\pgfsetdash{}{0pt}
\pgfsetdash{}{0pt}
\pgfsetbuttcap
{
\definecolor{dialinecolor}{rgb}{0.000000, 0.000000, 0.000000}
\pgfsetfillcolor{dialinecolor}
% was here!!!
\definecolor{dialinecolor}{rgb}{0.000000, 0.000000, 0.000000}
\pgfsetstrokecolor{dialinecolor}
\draw (33.851616\du,30.694834\du)--(33.836131\du,32.366566\du);
}
\pgfsetlinewidth{0.100000\du}
\pgfsetdash{}{0pt}
\pgfsetdash{}{0pt}
\pgfsetbuttcap
{
\definecolor{dialinecolor}{rgb}{0.000000, 0.000000, 0.000000}
\pgfsetfillcolor{dialinecolor}
% was here!!!
\definecolor{dialinecolor}{rgb}{0.000000, 0.000000, 0.000000}
\pgfsetstrokecolor{dialinecolor}
\draw (33.836944\du,35.601634\du)--(33.837766\du,37.261953\du);
}
\pgfsetlinewidth{0.100000\du}
\pgfsetdash{}{0pt}
\pgfsetdash{}{0pt}
\pgfsetbuttcap
{
\definecolor{dialinecolor}{rgb}{0.000000, 0.000000, 0.000000}
\pgfsetfillcolor{dialinecolor}
% was here!!!
\definecolor{dialinecolor}{rgb}{0.000000, 0.000000, 0.000000}
\pgfsetstrokecolor{dialinecolor}
\draw (33.840524\du,21.336824\du)--(33.848922\du,22.670264\du);
}
\definecolor{dialinecolor}{rgb}{1.000000, 1.000000, 1.000000}
\pgfsetfillcolor{dialinecolor}
\pgfpathellipse{\pgfpoint{29.527972\du}{24.269741\du}}{\pgfpoint{1.503364\du}{0\du}}{\pgfpoint{0\du}{1.501682\du}}
\pgfusepath{fill}
\pgfsetlinewidth{0.100000\du}
\pgfsetdash{}{0pt}
\pgfsetdash{}{0pt}
\pgfsetmiterjoin
\definecolor{dialinecolor}{rgb}{0.000000, 0.000000, 0.000000}
\pgfsetstrokecolor{dialinecolor}
\pgfpathellipse{\pgfpoint{29.527972\du}{24.269741\du}}{\pgfpoint{1.503364\du}{0\du}}{\pgfpoint{0\du}{1.501682\du}}
\pgfusepath{stroke}
% setfont left to latex
\definecolor{dialinecolor}{rgb}{0.000000, 0.000000, 0.000000}
\pgfsetstrokecolor{dialinecolor}
\node at (29.527972\du,24.464741\du){P};
\pgfsetlinewidth{0.100000\du}
\pgfsetdash{}{0pt}
\pgfsetdash{}{0pt}
\pgfsetbuttcap
{
\definecolor{dialinecolor}{rgb}{0.000000, 0.000000, 0.000000}
\pgfsetfillcolor{dialinecolor}
% was here!!!
\definecolor{dialinecolor}{rgb}{0.000000, 0.000000, 0.000000}
\pgfsetstrokecolor{dialinecolor}
\draw (31.031336\du,24.269741\du)--(32.270596\du,24.262753\du);
}
\definecolor{dialinecolor}{rgb}{1.000000, 1.000000, 1.000000}
\pgfsetfillcolor{dialinecolor}
\pgfpathellipse{\pgfpoint{25.265564\du}{24.263343\du}}{\pgfpoint{1.503364\du}{0\du}}{\pgfpoint{0\du}{1.501682\du}}
\pgfusepath{fill}
\pgfsetlinewidth{0.100000\du}
\pgfsetdash{}{0pt}
\pgfsetdash{}{0pt}
\pgfsetmiterjoin
\definecolor{dialinecolor}{rgb}{0.000000, 0.000000, 0.000000}
\pgfsetstrokecolor{dialinecolor}
\pgfpathellipse{\pgfpoint{25.265564\du}{24.263343\du}}{\pgfpoint{1.503364\du}{0\du}}{\pgfpoint{0\du}{1.501682\du}}
\pgfusepath{stroke}
% setfont left to latex
\definecolor{dialinecolor}{rgb}{0.000000, 0.000000, 0.000000}
\pgfsetstrokecolor{dialinecolor}
\node at (25.265564\du,24.458343\du){P};
\pgfsetlinewidth{0.100000\du}
\pgfsetdash{}{0pt}
\pgfsetdash{}{0pt}
\pgfsetbuttcap
{
\definecolor{dialinecolor}{rgb}{0.000000, 0.000000, 0.000000}
\pgfsetfillcolor{dialinecolor}
% was here!!!
\definecolor{dialinecolor}{rgb}{0.000000, 0.000000, 0.000000}
\pgfsetstrokecolor{dialinecolor}
\draw (26.768928\du,24.263343\du)--(28.024608\du,24.269741\du);
}
\definecolor{dialinecolor}{rgb}{1.000000, 1.000000, 1.000000}
\pgfsetfillcolor{dialinecolor}
\pgfpathellipse{\pgfpoint{29.515205\du}{29.099873\du}}{\pgfpoint{1.503364\du}{0\du}}{\pgfpoint{0\du}{1.501682\du}}
\pgfusepath{fill}
\pgfsetlinewidth{0.100000\du}
\pgfsetdash{}{0pt}
\pgfsetdash{}{0pt}
\pgfsetmiterjoin
\definecolor{dialinecolor}{rgb}{0.000000, 0.000000, 0.000000}
\pgfsetstrokecolor{dialinecolor}
\pgfpathellipse{\pgfpoint{29.515205\du}{29.099873\du}}{\pgfpoint{1.503364\du}{0\du}}{\pgfpoint{0\du}{1.501682\du}}
\pgfusepath{stroke}
% setfont left to latex
\definecolor{dialinecolor}{rgb}{0.000000, 0.000000, 0.000000}
\pgfsetstrokecolor{dialinecolor}
\node at (29.515205\du,29.294873\du){P};
\pgfsetlinewidth{0.100000\du}
\pgfsetdash{}{0pt}
\pgfsetdash{}{0pt}
\pgfsetbuttcap
{
\definecolor{dialinecolor}{rgb}{0.000000, 0.000000, 0.000000}
\pgfsetfillcolor{dialinecolor}
% was here!!!
\definecolor{dialinecolor}{rgb}{0.000000, 0.000000, 0.000000}
\pgfsetstrokecolor{dialinecolor}
\draw (31.018569\du,29.099873\du)--(32.273289\du,29.102345\du);
}
\pgfsetlinewidth{0.100000\du}
\pgfsetdash{}{0pt}
\pgfsetdash{}{0pt}
\pgfsetbuttcap
{
\definecolor{dialinecolor}{rgb}{0.000000, 0.000000, 0.000000}
\pgfsetfillcolor{dialinecolor}
% was here!!!
\definecolor{dialinecolor}{rgb}{0.000000, 0.000000, 0.000000}
\pgfsetstrokecolor{dialinecolor}
\draw (23.762200\du,24.263343\du)--(22.176036\du,24.265502\du);
}
\pgfsetlinewidth{0.100000\du}
\pgfsetdash{}{0pt}
\pgfsetdash{}{0pt}
\pgfsetbuttcap
{
\definecolor{dialinecolor}{rgb}{0.000000, 0.000000, 0.000000}
\pgfsetfillcolor{dialinecolor}
% was here!!!
\definecolor{dialinecolor}{rgb}{0.000000, 0.000000, 0.000000}
\pgfsetstrokecolor{dialinecolor}
\draw (27.961534\du,29.099956\du)--(22.202552\du,29.100265\du);
}
\pgfsetlinewidth{0.100000\du}
\pgfsetdash{}{0pt}
\pgfsetdash{}{0pt}
\pgfsetbuttcap
{
\definecolor{dialinecolor}{rgb}{0.000000, 0.000000, 0.000000}
\pgfsetfillcolor{dialinecolor}
% was here!!!
\definecolor{dialinecolor}{rgb}{0.000000, 0.000000, 0.000000}
\pgfsetstrokecolor{dialinecolor}
\draw (32.257805\du,33.959055\du)--(22.335132\du,33.961543\du);
}
\end{tikzpicture}

\caption{LHS of the rule for n = 3}
\end{center}
\end{figure}
\subsubsection*{Right-hand side}
\begin{figure}[H]
\begin{center}
% Graphic for TeX using PGF
% Title: /home/yiannis/Desktop/cw-custom/diafiles/hor2.dia
% Creator: Dia v0.97.2
% CreationDate: Sat Mar  2 03:30:31 2013
% For: yiannis
% \usepackage{tikz}
% The following commands are not supported in PSTricks at present
% We define them conditionally, so when they are implemented,
% this pgf file will use them.
\ifx\du\undefined
  \newlength{\du}
\fi
\setlength{\du}{15\unitlength}
\begin{tikzpicture}[thick, scale=0.4]
\pgftransformxscale{1.000000}
\pgftransformyscale{-1.000000}
\definecolor{dialinecolor}{rgb}{0.000000, 0.000000, 0.000000}
\pgfsetstrokecolor{dialinecolor}
\definecolor{dialinecolor}{rgb}{1.000000, 1.000000, 1.000000}
\pgfsetfillcolor{dialinecolor}
\definecolor{dialinecolor}{rgb}{1.000000, 1.000000, 1.000000}
\pgfsetfillcolor{dialinecolor}
\pgfpathellipse{\pgfpoint{52.322335\du}{12.266573\du}}{\pgfpoint{1.503364\du}{0\du}}{\pgfpoint{0\du}{1.501682\du}}
\pgfusepath{fill}
\pgfsetlinewidth{0.100000\du}
\pgfsetdash{}{0pt}
\pgfsetdash{}{0pt}
\pgfsetmiterjoin
\definecolor{dialinecolor}{rgb}{0.000000, 0.000000, 0.000000}
\pgfsetstrokecolor{dialinecolor}
\pgfpathellipse{\pgfpoint{52.322335\du}{12.266573\du}}{\pgfpoint{1.503364\du}{0\du}}{\pgfpoint{0\du}{1.501682\du}}
\pgfusepath{stroke}
% setfont left to latex
\definecolor{dialinecolor}{rgb}{0.000000, 0.000000, 0.000000}
\pgfsetstrokecolor{dialinecolor}
\node at (52.322335\du,12.461573\du){Q};
\pgfsetlinewidth{0.100000\du}
\pgfsetdash{}{0pt}
\pgfsetdash{}{0pt}
\pgfsetbuttcap
{
\definecolor{dialinecolor}{rgb}{0.000000, 0.000000, 0.000000}
\pgfsetfillcolor{dialinecolor}
% was here!!!
\definecolor{dialinecolor}{rgb}{0.000000, 0.000000, 0.000000}
\pgfsetstrokecolor{dialinecolor}
\draw (52.325174\du,9.294795\du)--(52.322335\du,10.764891\du);
}
\definecolor{dialinecolor}{rgb}{1.000000, 1.000000, 1.000000}
\pgfsetfillcolor{dialinecolor}
\fill (50.752407\du,15.101694\du)--(50.752407\du,18.286673\du)--(53.909059\du,18.286673\du)--(53.909059\du,15.101694\du)--cycle;
\pgfsetlinewidth{0.100000\du}
\pgfsetdash{}{0pt}
\pgfsetdash{}{0pt}
\pgfsetmiterjoin
\definecolor{dialinecolor}{rgb}{0.000000, 0.000000, 0.000000}
\pgfsetstrokecolor{dialinecolor}
\draw (50.752407\du,15.101694\du)--(50.752407\du,18.286673\du)--(53.909059\du,18.286673\du)--(53.909059\du,15.101694\du)--cycle;
% setfont left to latex
\definecolor{dialinecolor}{rgb}{0.000000, 0.000000, 0.000000}
\pgfsetstrokecolor{dialinecolor}
\node at (52.330733\du,16.889184\du){R};
\pgfsetlinewidth{0.100000\du}
\pgfsetdash{}{0pt}
\pgfsetdash{}{0pt}
\pgfsetbuttcap
{
\definecolor{dialinecolor}{rgb}{0.000000, 0.000000, 0.000000}
\pgfsetfillcolor{dialinecolor}
% was here!!!
\definecolor{dialinecolor}{rgb}{0.000000, 0.000000, 0.000000}
\pgfsetstrokecolor{dialinecolor}
\draw (52.322335\du,13.768255\du)--(52.330733\du,15.101694\du);
}
\pgfsetlinewidth{0.100000\du}
\pgfsetdash{}{0pt}
\pgfsetdash{}{0pt}
\pgfsetbuttcap
{
\definecolor{dialinecolor}{rgb}{0.000000, 0.000000, 0.000000}
\pgfsetfillcolor{dialinecolor}
% was here!!!
\definecolor{dialinecolor}{rgb}{0.000000, 0.000000, 0.000000}
\pgfsetstrokecolor{dialinecolor}
\draw (47.551983\du,16.674912\du)--(50.752407\du,16.694184\du);
}
\definecolor{dialinecolor}{rgb}{1.000000, 1.000000, 1.000000}
\pgfsetfillcolor{dialinecolor}
\pgfpathellipse{\pgfpoint{52.323204\du}{21.255443\du}}{\pgfpoint{1.503364\du}{0\du}}{\pgfpoint{0\du}{1.501682\du}}
\pgfusepath{fill}
\pgfsetlinewidth{0.100000\du}
\pgfsetdash{}{0pt}
\pgfsetdash{}{0pt}
\pgfsetmiterjoin
\definecolor{dialinecolor}{rgb}{0.000000, 0.000000, 0.000000}
\pgfsetstrokecolor{dialinecolor}
\pgfpathellipse{\pgfpoint{52.323204\du}{21.255443\du}}{\pgfpoint{1.503364\du}{0\du}}{\pgfpoint{0\du}{1.501682\du}}
\pgfusepath{stroke}
% setfont left to latex
\definecolor{dialinecolor}{rgb}{0.000000, 0.000000, 0.000000}
\pgfsetstrokecolor{dialinecolor}
\node at (52.323204\du,21.450443\du){Q};
\pgfsetlinewidth{0.100000\du}
\pgfsetdash{}{0pt}
\pgfsetdash{}{0pt}
\pgfsetbuttcap
{
\definecolor{dialinecolor}{rgb}{0.000000, 0.000000, 0.000000}
\pgfsetfillcolor{dialinecolor}
% was here!!!
\definecolor{dialinecolor}{rgb}{0.000000, 0.000000, 0.000000}
\pgfsetstrokecolor{dialinecolor}
\draw (52.330733\du,18.286673\du)--(52.323204\du,19.753761\du);
}
\definecolor{dialinecolor}{rgb}{1.000000, 1.000000, 1.000000}
\pgfsetfillcolor{dialinecolor}
\fill (50.739645\du,24.090565\du)--(50.739645\du,27.275543\du)--(53.896297\du,27.275543\du)--(53.896297\du,24.090565\du)--cycle;
\pgfsetlinewidth{0.100000\du}
\pgfsetdash{}{0pt}
\pgfsetdash{}{0pt}
\pgfsetmiterjoin
\definecolor{dialinecolor}{rgb}{0.000000, 0.000000, 0.000000}
\pgfsetstrokecolor{dialinecolor}
\draw (50.739645\du,24.090565\du)--(50.739645\du,27.275543\du)--(53.896297\du,27.275543\du)--(53.896297\du,24.090565\du)--cycle;
% setfont left to latex
\definecolor{dialinecolor}{rgb}{0.000000, 0.000000, 0.000000}
\pgfsetstrokecolor{dialinecolor}
\node at (52.317971\du,25.878054\du){R};
\pgfsetlinewidth{0.100000\du}
\pgfsetdash{}{0pt}
\pgfsetdash{}{0pt}
\pgfsetbuttcap
{
\definecolor{dialinecolor}{rgb}{0.000000, 0.000000, 0.000000}
\pgfsetfillcolor{dialinecolor}
% was here!!!
\definecolor{dialinecolor}{rgb}{0.000000, 0.000000, 0.000000}
\pgfsetstrokecolor{dialinecolor}
\draw (52.323204\du,22.757125\du)--(52.317971\du,24.090565\du);
}
\pgfsetlinewidth{0.100000\du}
\pgfsetdash{}{0pt}
\pgfsetdash{}{0pt}
\pgfsetbuttcap
{
\definecolor{dialinecolor}{rgb}{0.000000, 0.000000, 0.000000}
\pgfsetfillcolor{dialinecolor}
% was here!!!
\definecolor{dialinecolor}{rgb}{0.000000, 0.000000, 0.000000}
\pgfsetstrokecolor{dialinecolor}
\draw (47.526983\du,25.674704\du)--(50.739645\du,25.683054\du);
}
\definecolor{dialinecolor}{rgb}{1.000000, 1.000000, 1.000000}
\pgfsetfillcolor{dialinecolor}
\pgfpathellipse{\pgfpoint{52.317354\du}{30.232341\du}}{\pgfpoint{1.503364\du}{0\du}}{\pgfpoint{0\du}{1.501682\du}}
\pgfusepath{fill}
\pgfsetlinewidth{0.100000\du}
\pgfsetdash{}{0pt}
\pgfsetdash{}{0pt}
\pgfsetmiterjoin
\definecolor{dialinecolor}{rgb}{0.000000, 0.000000, 0.000000}
\pgfsetstrokecolor{dialinecolor}
\pgfpathellipse{\pgfpoint{52.317354\du}{30.232341\du}}{\pgfpoint{1.503364\du}{0\du}}{\pgfpoint{0\du}{1.501682\du}}
\pgfusepath{stroke}
% setfont left to latex
\definecolor{dialinecolor}{rgb}{0.000000, 0.000000, 0.000000}
\pgfsetstrokecolor{dialinecolor}
\node at (52.317354\du,30.427341\du){Q};
\pgfsetlinewidth{0.100000\du}
\pgfsetdash{}{0pt}
\pgfsetdash{}{0pt}
\pgfsetbuttcap
{
\definecolor{dialinecolor}{rgb}{0.000000, 0.000000, 0.000000}
\pgfsetfillcolor{dialinecolor}
% was here!!!
\definecolor{dialinecolor}{rgb}{0.000000, 0.000000, 0.000000}
\pgfsetstrokecolor{dialinecolor}
\draw (52.317971\du,27.275543\du)--(52.317354\du,28.730660\du);
}
\definecolor{dialinecolor}{rgb}{1.000000, 1.000000, 1.000000}
\pgfsetfillcolor{dialinecolor}
\fill (50.747426\du,33.067463\du)--(50.747426\du,36.252442\du)--(53.904079\du,36.252442\du)--(53.904079\du,33.067463\du)--cycle;
\pgfsetlinewidth{0.100000\du}
\pgfsetdash{}{0pt}
\pgfsetdash{}{0pt}
\pgfsetmiterjoin
\definecolor{dialinecolor}{rgb}{0.000000, 0.000000, 0.000000}
\pgfsetstrokecolor{dialinecolor}
\draw (50.747426\du,33.067463\du)--(50.747426\du,36.252442\du)--(53.904079\du,36.252442\du)--(53.904079\du,33.067463\du)--cycle;
% setfont left to latex
\definecolor{dialinecolor}{rgb}{0.000000, 0.000000, 0.000000}
\pgfsetstrokecolor{dialinecolor}
\node at (52.325752\du,34.854952\du){R};
\pgfsetlinewidth{0.100000\du}
\pgfsetdash{}{0pt}
\pgfsetdash{}{0pt}
\pgfsetbuttcap
{
\definecolor{dialinecolor}{rgb}{0.000000, 0.000000, 0.000000}
\pgfsetfillcolor{dialinecolor}
% was here!!!
\definecolor{dialinecolor}{rgb}{0.000000, 0.000000, 0.000000}
\pgfsetstrokecolor{dialinecolor}
\draw (52.317354\du,31.734023\du)--(52.325752\du,33.067463\du);
}
\pgfsetlinewidth{0.100000\du}
\pgfsetdash{}{0pt}
\pgfsetdash{}{0pt}
\pgfsetbuttcap
{
\definecolor{dialinecolor}{rgb}{0.000000, 0.000000, 0.000000}
\pgfsetfillcolor{dialinecolor}
% was here!!!
\definecolor{dialinecolor}{rgb}{0.000000, 0.000000, 0.000000}
\pgfsetstrokecolor{dialinecolor}
\draw (47.551983\du,34.674495\du)--(50.747426\du,34.659952\du);
}
\pgfsetlinewidth{0.100000\du}
\pgfsetdash{}{0pt}
\pgfsetdash{}{0pt}
\pgfsetbuttcap
{
\definecolor{dialinecolor}{rgb}{0.000000, 0.000000, 0.000000}
\pgfsetfillcolor{dialinecolor}
% was here!!!
\definecolor{dialinecolor}{rgb}{0.000000, 0.000000, 0.000000}
\pgfsetstrokecolor{dialinecolor}
\draw (52.312846\du,36.302668\du)--(52.297848\du,38.211582\du);
}
\end{tikzpicture}

\caption{RHS of the rule for n = 3}
\end{center}
\end{figure}
\subsection*{(d) Polynomial Evaluation}
R stands for the add operation (addition), P and Q both stand for multiplication by a constant (let this constant be x).
For the given coefficients $a_0,a_1,a_2,a_3$, the circuit will be the following.
\begin{figure}[H]
\begin{center}
% Graphic for TeX using PGF
% Title: /home/yiannis/Desktop/cw-custom/diafiles/hormult.dia
% Creator: Dia v0.97.2
% CreationDate: Sat Mar  2 03:55:05 2013
% For: yiannis
% \usepackage{tikz}
% The following commands are not supported in PSTricks at present
% We define them conditionally, so when they are implemented,
% this pgf file will use them.
\ifx\du\undefined
  \newlength{\du}
\fi
\setlength{\du}{15\unitlength}
\begin{tikzpicture}
\pgftransformxscale{1.000000}
\pgftransformyscale{-1.000000}
\definecolor{dialinecolor}{rgb}{0.000000, 0.000000, 0.000000}
\pgfsetstrokecolor{dialinecolor}
\definecolor{dialinecolor}{rgb}{1.000000, 1.000000, 1.000000}
\pgfsetfillcolor{dialinecolor}
\definecolor{dialinecolor}{rgb}{1.000000, 1.000000, 1.000000}
\pgfsetfillcolor{dialinecolor}
\pgfpathellipse{\pgfpoint{52.322335\du}{12.266573\du}}{\pgfpoint{1.503364\du}{0\du}}{\pgfpoint{0\du}{1.501682\du}}
\pgfusepath{fill}
\pgfsetlinewidth{0.100000\du}
\pgfsetdash{}{0pt}
\pgfsetdash{}{0pt}
\pgfsetmiterjoin
\definecolor{dialinecolor}{rgb}{0.000000, 0.000000, 0.000000}
\pgfsetstrokecolor{dialinecolor}
\pgfpathellipse{\pgfpoint{52.322335\du}{12.266573\du}}{\pgfpoint{1.503364\du}{0\du}}{\pgfpoint{0\du}{1.501682\du}}
\pgfusepath{stroke}
% setfont left to latex
\definecolor{dialinecolor}{rgb}{0.000000, 0.000000, 0.000000}
\pgfsetstrokecolor{dialinecolor}
\node at (52.322335\du,12.461573\du){Q};
\pgfsetlinewidth{0.100000\du}
\pgfsetdash{}{0pt}
\pgfsetdash{}{0pt}
\pgfsetbuttcap
{
\definecolor{dialinecolor}{rgb}{0.000000, 0.000000, 0.000000}
\pgfsetfillcolor{dialinecolor}
% was here!!!
\definecolor{dialinecolor}{rgb}{0.000000, 0.000000, 0.000000}
\pgfsetstrokecolor{dialinecolor}
\draw (52.340407\du,8.481406\du)--(52.331973\du,10.764891\du);
}
\definecolor{dialinecolor}{rgb}{1.000000, 1.000000, 1.000000}
\pgfsetfillcolor{dialinecolor}
\fill (50.752407\du,15.101694\du)--(50.752407\du,18.286673\du)--(53.909059\du,18.286673\du)--(53.909059\du,15.101694\du)--cycle;
\pgfsetlinewidth{0.100000\du}
\pgfsetdash{}{0pt}
\pgfsetdash{}{0pt}
\pgfsetmiterjoin
\definecolor{dialinecolor}{rgb}{0.000000, 0.000000, 0.000000}
\pgfsetstrokecolor{dialinecolor}
\draw (50.752407\du,15.101694\du)--(50.752407\du,18.286673\du)--(53.909059\du,18.286673\du)--(53.909059\du,15.101694\du)--cycle;
% setfont left to latex
\definecolor{dialinecolor}{rgb}{0.000000, 0.000000, 0.000000}
\pgfsetstrokecolor{dialinecolor}
\node at (52.330733\du,16.889184\du){R};
\pgfsetlinewidth{0.100000\du}
\pgfsetdash{}{0pt}
\pgfsetdash{}{0pt}
\pgfsetbuttcap
{
\definecolor{dialinecolor}{rgb}{0.000000, 0.000000, 0.000000}
\pgfsetfillcolor{dialinecolor}
% was here!!!
\definecolor{dialinecolor}{rgb}{0.000000, 0.000000, 0.000000}
\pgfsetstrokecolor{dialinecolor}
\draw (52.322335\du,13.768255\du)--(52.330733\du,15.101694\du);
}
\pgfsetlinewidth{0.100000\du}
\pgfsetdash{}{0pt}
\pgfsetdash{}{0pt}
\pgfsetbuttcap
{
\definecolor{dialinecolor}{rgb}{0.000000, 0.000000, 0.000000}
\pgfsetfillcolor{dialinecolor}
% was here!!!
\definecolor{dialinecolor}{rgb}{0.000000, 0.000000, 0.000000}
\pgfsetstrokecolor{dialinecolor}
\draw (46.017295\du,16.669353\du)--(50.752407\du,16.694184\du);
}
\definecolor{dialinecolor}{rgb}{1.000000, 1.000000, 1.000000}
\pgfsetfillcolor{dialinecolor}
\pgfpathellipse{\pgfpoint{52.323204\du}{21.255443\du}}{\pgfpoint{1.503364\du}{0\du}}{\pgfpoint{0\du}{1.501682\du}}
\pgfusepath{fill}
\pgfsetlinewidth{0.100000\du}
\pgfsetdash{}{0pt}
\pgfsetdash{}{0pt}
\pgfsetmiterjoin
\definecolor{dialinecolor}{rgb}{0.000000, 0.000000, 0.000000}
\pgfsetstrokecolor{dialinecolor}
\pgfpathellipse{\pgfpoint{52.323204\du}{21.255443\du}}{\pgfpoint{1.503364\du}{0\du}}{\pgfpoint{0\du}{1.501682\du}}
\pgfusepath{stroke}
% setfont left to latex
\definecolor{dialinecolor}{rgb}{0.000000, 0.000000, 0.000000}
\pgfsetstrokecolor{dialinecolor}
\node at (52.323204\du,21.450443\du){Q};
\pgfsetlinewidth{0.100000\du}
\pgfsetdash{}{0pt}
\pgfsetdash{}{0pt}
\pgfsetbuttcap
{
\definecolor{dialinecolor}{rgb}{0.000000, 0.000000, 0.000000}
\pgfsetfillcolor{dialinecolor}
% was here!!!
\definecolor{dialinecolor}{rgb}{0.000000, 0.000000, 0.000000}
\pgfsetstrokecolor{dialinecolor}
\draw (52.330733\du,18.286673\du)--(52.323204\du,19.753761\du);
}
\definecolor{dialinecolor}{rgb}{1.000000, 1.000000, 1.000000}
\pgfsetfillcolor{dialinecolor}
\fill (50.739645\du,24.090565\du)--(50.739645\du,27.275543\du)--(53.896297\du,27.275543\du)--(53.896297\du,24.090565\du)--cycle;
\pgfsetlinewidth{0.100000\du}
\pgfsetdash{}{0pt}
\pgfsetdash{}{0pt}
\pgfsetmiterjoin
\definecolor{dialinecolor}{rgb}{0.000000, 0.000000, 0.000000}
\pgfsetstrokecolor{dialinecolor}
\draw (50.739645\du,24.090565\du)--(50.739645\du,27.275543\du)--(53.896297\du,27.275543\du)--(53.896297\du,24.090565\du)--cycle;
% setfont left to latex
\definecolor{dialinecolor}{rgb}{0.000000, 0.000000, 0.000000}
\pgfsetstrokecolor{dialinecolor}
\node at (52.317971\du,25.878054\du){R};
\pgfsetlinewidth{0.100000\du}
\pgfsetdash{}{0pt}
\pgfsetdash{}{0pt}
\pgfsetbuttcap
{
\definecolor{dialinecolor}{rgb}{0.000000, 0.000000, 0.000000}
\pgfsetfillcolor{dialinecolor}
% was here!!!
\definecolor{dialinecolor}{rgb}{0.000000, 0.000000, 0.000000}
\pgfsetstrokecolor{dialinecolor}
\draw (52.323204\du,22.757125\du)--(52.317971\du,24.090565\du);
}
\pgfsetlinewidth{0.100000\du}
\pgfsetdash{}{0pt}
\pgfsetdash{}{0pt}
\pgfsetbuttcap
{
\definecolor{dialinecolor}{rgb}{0.000000, 0.000000, 0.000000}
\pgfsetfillcolor{dialinecolor}
% was here!!!
\definecolor{dialinecolor}{rgb}{0.000000, 0.000000, 0.000000}
\pgfsetstrokecolor{dialinecolor}
\draw (46.005900\du,25.668589\du)--(50.739645\du,25.683054\du);
}
\definecolor{dialinecolor}{rgb}{1.000000, 1.000000, 1.000000}
\pgfsetfillcolor{dialinecolor}
\pgfpathellipse{\pgfpoint{52.317354\du}{30.232341\du}}{\pgfpoint{1.503364\du}{0\du}}{\pgfpoint{0\du}{1.501682\du}}
\pgfusepath{fill}
\pgfsetlinewidth{0.100000\du}
\pgfsetdash{}{0pt}
\pgfsetdash{}{0pt}
\pgfsetmiterjoin
\definecolor{dialinecolor}{rgb}{0.000000, 0.000000, 0.000000}
\pgfsetstrokecolor{dialinecolor}
\pgfpathellipse{\pgfpoint{52.317354\du}{30.232341\du}}{\pgfpoint{1.503364\du}{0\du}}{\pgfpoint{0\du}{1.501682\du}}
\pgfusepath{stroke}
% setfont left to latex
\definecolor{dialinecolor}{rgb}{0.000000, 0.000000, 0.000000}
\pgfsetstrokecolor{dialinecolor}
\node at (52.317354\du,30.427341\du){Q};
\pgfsetlinewidth{0.100000\du}
\pgfsetdash{}{0pt}
\pgfsetdash{}{0pt}
\pgfsetbuttcap
{
\definecolor{dialinecolor}{rgb}{0.000000, 0.000000, 0.000000}
\pgfsetfillcolor{dialinecolor}
% was here!!!
\definecolor{dialinecolor}{rgb}{0.000000, 0.000000, 0.000000}
\pgfsetstrokecolor{dialinecolor}
\draw (52.317971\du,27.275543\du)--(52.317354\du,28.730660\du);
}
\definecolor{dialinecolor}{rgb}{1.000000, 1.000000, 1.000000}
\pgfsetfillcolor{dialinecolor}
\fill (50.747426\du,33.067463\du)--(50.747426\du,36.252442\du)--(53.904079\du,36.252442\du)--(53.904079\du,33.067463\du)--cycle;
\pgfsetlinewidth{0.100000\du}
\pgfsetdash{}{0pt}
\pgfsetdash{}{0pt}
\pgfsetmiterjoin
\definecolor{dialinecolor}{rgb}{0.000000, 0.000000, 0.000000}
\pgfsetstrokecolor{dialinecolor}
\draw (50.747426\du,33.067463\du)--(50.747426\du,36.252442\du)--(53.904079\du,36.252442\du)--(53.904079\du,33.067463\du)--cycle;
% setfont left to latex
\definecolor{dialinecolor}{rgb}{0.000000, 0.000000, 0.000000}
\pgfsetstrokecolor{dialinecolor}
\node at (52.325752\du,34.854952\du){R};
\pgfsetlinewidth{0.100000\du}
\pgfsetdash{}{0pt}
\pgfsetdash{}{0pt}
\pgfsetbuttcap
{
\definecolor{dialinecolor}{rgb}{0.000000, 0.000000, 0.000000}
\pgfsetfillcolor{dialinecolor}
% was here!!!
\definecolor{dialinecolor}{rgb}{0.000000, 0.000000, 0.000000}
\pgfsetstrokecolor{dialinecolor}
\draw (52.317354\du,31.734023\du)--(52.325752\du,33.067463\du);
}
\pgfsetlinewidth{0.100000\du}
\pgfsetdash{}{0pt}
\pgfsetdash{}{0pt}
\pgfsetbuttcap
{
\definecolor{dialinecolor}{rgb}{0.000000, 0.000000, 0.000000}
\pgfsetfillcolor{dialinecolor}
% was here!!!
\definecolor{dialinecolor}{rgb}{0.000000, 0.000000, 0.000000}
\pgfsetstrokecolor{dialinecolor}
\draw (45.913164\du,34.631191\du)--(50.747426\du,34.659952\du);
}
\pgfsetlinewidth{0.100000\du}
\pgfsetdash{}{0pt}
\pgfsetdash{}{0pt}
\pgfsetbuttcap
{
\definecolor{dialinecolor}{rgb}{0.000000, 0.000000, 0.000000}
\pgfsetfillcolor{dialinecolor}
% was here!!!
\definecolor{dialinecolor}{rgb}{0.000000, 0.000000, 0.000000}
\pgfsetstrokecolor{dialinecolor}
\draw (52.312846\du,36.302668\du)--(52.297848\du,38.211582\du);
}
\definecolor{dialinecolor}{rgb}{1.000000, 1.000000, 1.000000}
\pgfsetfillcolor{dialinecolor}
\fill (43.117295\du,15.219353\du)--(43.117295\du,18.119353\du)--(46.017295\du,18.119353\du)--(46.017295\du,15.219353\du)--cycle;
\pgfsetlinewidth{0.100000\du}
\pgfsetdash{{\pgflinewidth}{0.200000\du}}{0cm}
\pgfsetdash{{\pgflinewidth}{0.000200\du}}{0cm}
\pgfsetmiterjoin
\definecolor{dialinecolor}{rgb}{1.000000, 1.000000, 1.000000}
\pgfsetstrokecolor{dialinecolor}
\draw (43.117295\du,15.219353\du)--(43.117295\du,18.119353\du)--(46.017295\du,18.119353\du)--(46.017295\du,15.219353\du)--cycle;
% setfont left to latex
\definecolor{dialinecolor}{rgb}{0.000000, 0.000000, 0.000000}
\pgfsetstrokecolor{dialinecolor}
\node at (44.567295\du,16.864353\du){a2};
\definecolor{dialinecolor}{rgb}{1.000000, 1.000000, 1.000000}
\pgfsetfillcolor{dialinecolor}
\fill (43.057110\du,24.214009\du)--(43.057110\du,27.114009\du)--(45.957110\du,27.114009\du)--(45.957110\du,24.214009\du)--cycle;
\pgfsetlinewidth{0.100000\du}
\pgfsetdash{{\pgflinewidth}{0.000200\du}}{0cm}
\pgfsetdash{{\pgflinewidth}{0.000200\du}}{0cm}
\pgfsetmiterjoin
\definecolor{dialinecolor}{rgb}{1.000000, 1.000000, 1.000000}
\pgfsetstrokecolor{dialinecolor}
\draw (43.057110\du,24.214009\du)--(43.057110\du,27.114009\du)--(45.957110\du,27.114009\du)--(45.957110\du,24.214009\du)--cycle;
% setfont left to latex
\definecolor{dialinecolor}{rgb}{0.000000, 0.000000, 0.000000}
\pgfsetstrokecolor{dialinecolor}
\node at (44.507110\du,25.859009\du){a1};
\definecolor{dialinecolor}{rgb}{1.000000, 1.000000, 1.000000}
\pgfsetfillcolor{dialinecolor}
\fill (43.013164\du,33.181191\du)--(43.013164\du,36.081191\du)--(45.913164\du,36.081191\du)--(45.913164\du,33.181191\du)--cycle;
\pgfsetlinewidth{0.100000\du}
\pgfsetdash{{\pgflinewidth}{0.000200\du}}{0cm}
\pgfsetdash{{\pgflinewidth}{0.000200\du}}{0cm}
\pgfsetmiterjoin
\definecolor{dialinecolor}{rgb}{1.000000, 1.000000, 1.000000}
\pgfsetstrokecolor{dialinecolor}
\draw (43.013164\du,33.181191\du)--(43.013164\du,36.081191\du)--(45.913164\du,36.081191\du)--(45.913164\du,33.181191\du)--cycle;
% setfont left to latex
\definecolor{dialinecolor}{rgb}{0.000000, 0.000000, 0.000000}
\pgfsetstrokecolor{dialinecolor}
\node at (44.463164\du,34.826191\du){a0};
\definecolor{dialinecolor}{rgb}{1.000000, 1.000000, 1.000000}
\pgfsetfillcolor{dialinecolor}
\fill (50.890407\du,5.581406\du)--(50.890407\du,8.481406\du)--(53.790407\du,8.481406\du)--(53.790407\du,5.581406\du)--cycle;
\pgfsetlinewidth{0.100000\du}
\pgfsetdash{{\pgflinewidth}{0.000200\du}}{0cm}
\pgfsetdash{{\pgflinewidth}{0.000200\du}}{0cm}
\pgfsetmiterjoin
\definecolor{dialinecolor}{rgb}{1.000000, 1.000000, 1.000000}
\pgfsetstrokecolor{dialinecolor}
\draw (50.890407\du,5.581406\du)--(50.890407\du,8.481406\du)--(53.790407\du,8.481406\du)--(53.790407\du,5.581406\du)--cycle;
% setfont left to latex
\definecolor{dialinecolor}{rgb}{0.000000, 0.000000, 0.000000}
\pgfsetstrokecolor{dialinecolor}
\node at (52.340407\du,7.226406\du){a3};
\end{tikzpicture}

\caption{The optimised circuit as adjusted for polynomial evaluation}
\end{center}
\end{figure}
and the simulation written in Ruby: (x should be replace by the required number)
\begin{Verbatim}
INCLUDE "prelude.rby".
multc n = pi1^~1;snd n;mult.
Q = multc 'x'.
R = add.
POL n = rdr n (snd Q; R).
current = POL 3.
\end{Verbatim}
run with \verb|re "a_0 a_1 a_2 a_3"| produces the following output:
\begin{Verbatim}
Simulation start :

    0 - <<a_0,a_1,a_2>,a_3> ~ (a_0 + ((a_1 + ((a_2 + (a_3 * x)) * x)) * x))

Simulation end :
\end{Verbatim}
\section*{Question 4}
\subsection*{(a) Non-recursive definition of $btree_3$ and its type}
The non-recursive definition of $btree_3$ R is:
\begin{Verbatim}
 btree 3 R = [[R,R];R,[R,R];R];R.
\end{Verbatim}
and therefore its type is given by: \\[0.25cm]
$<<<X,X>,<X,X>>,<<X,X>,<X,X>>>\:  \sim \: X.$
\subsection*{(b) Fully pipelined timeless implementation of btree for timeless R}
\subsubsection*{Definition}
The system can be described by the following inductive equation: \\[0.25cm]
$pbtree_1 = R.$ \\[0.25cm]
$pbtree_{n+1} = [pbtree_n R,pbtree_n R];[D^n,D^n];R;AD^n.$
\subsubsection*{Proof by induction}
Here, we will try and prove that our equation is equivalent to the given.\\[0.5cm]
\textbf{Base Case:}  $pbtree_1 = btree_1$ (required to show)\\[0.25cm]
$pbtree_1$ \textbf{LHS} \\[0.25cm]
$=R$  (by definition of $pbtree$) \\[0.25cm]
$=btree_1 $ (by definition of $btree$) \textbf{RHS}
\\[0.5cm]
\textbf{Inductive Hypothesis:} $pbtree_k = btree_k $ \\[0.25cm]
We need to show that: $pbtree_{k+1} = btree_{k+1}.$ \\[0.25cm]
$pbtree_{k+1}$ \textbf{LHS} \\[0.25cm]
$=[pbtree_k R,pbtree_k R];[D^k,D^k];R;AD^k.$ (by the pbtree definition)\\[0.25cm]
$=[btree_k R,btree_k R];[D^k,D^k];R;AD^k.$ (replacing $pbtree_k$ with $btree_k$ by the hypothesis)\\[0.25cm]
$=[btree_k R,btree_k R];R.$ (replacing $[D^k,D^k];R;AD^k$ with R since R is timeless and they cancel out.)\\[0.25cm]
$=btree_{k+1}.$ (definition of btree) \textbf{RHS}
\subsubsection*{Symbolic simulation of a binary adder}
\begin{Verbatim}
INCLUDE "prelude.rby".
btree n R = 
	IF (n $eq 1) THEN
		(R) 
	ELSE 
		([btree (n-1) R, btree (n-1) R];[D^(n-1),D^(n-1)]; add ; (AD^(n-1))).
current = btree 3 add.
\end{Verbatim}
The results:
\begin{Verbatim}
re -s 3a b c d p q r s
Simulation start :
    0 - <<<a_0,b_0>,<c_0,d_0>>,<<p_0,q_0>,<r_0,s_0>>> 
	  ~ (((a_0 + b_0) + (c_0 + d_0)) + ((p_0 + q_0) + (r_0 + s_0)))
    1 - <<<a_1,b_1>,<c_1,d_1>>,<<p_1,q_1>,<r_1,s_1>>> ~ ?
    2 - <<<a_2,b_2>,<c_2,d_2>>,<<p_2,q_2>,<r_2,s_2>>> ~ ?
Simulation end :
\end{Verbatim}

\subsection*{(c) Change of type}
In order to understand the changes that need to occur in our definition let's consider $btree_3$. \\[0.5cm]
The transformation we wish to achieve is the following: \\[0.25cm]
$<X,X,X,X,X,X,X,X> \Rightarrow <<<X,X>,<X,X>>,<<X,X>,<X,X>>>$. \\[0.25cm]
which is equivallent to applying to the initial flat list the prelude function $half_4$ or $half_{2^(3-1)}$, and then to each half $half_2$. \\[0.25cm]
The obvious pattern is easily implementable in Ruby due to the functional nature of the language. The inductive definition of the equation is the following:
\\[0.5cm]
$btree_1 R = R$ \\[0.25cm]
$btree_n R = half_{2^{n-1}};[btree_{n-1} R, btree_{n-1} R];R.$ \\[0.25cm]
The source code of the implementation:
\begin{Verbatim}
INCLUDE "prelude.rby".
btree n R =
	IF (n $eq 1) THEN
		(R)
	ELSE
		(half (2 $exp (n-1));[btree (n-1) R, btree (n-1) R];R).
current = btree 3 add.
\end{Verbatim}
The result (sum 1 - 8):
\begin{Verbatim}
re 1 2 3 4 5 6 7 8
Simulation start :

    0 - <1,2,3,4,5,6,7,8> ~ 36

Simulation end :
\end{Verbatim}
The result (symbolic simulation):
\begin{Verbatim}
re a b c d p q r s
Simulation start :

    0 - <a,b,c,d,p,q,r,s> ~ (((a + b) + (c + d)) + ((p + q) + (r + s)))

Simulation end :
\end{Verbatim}


\end{document}
