\documentclass[a4paper,10pt]{article}
%\documentclass[a4paper,10pt]{scrartcl}

\usepackage[utf8]{inputenc}
\usepackage[margin=1in]{geometry}
\usepackage{tikz}
\usepackage{alltt}
\usepackage{fancyvrb}
\newcommand{\mytilde}{$\sim$}

\pdfinfo{%
  /Title    (Custom Computing: Assessed Coursework)
  /Author   (Ioannis Kassinopoulos)
  /Creator  (Ioannis Kassinopoulos)
  /Producer (Ioannis Kassinopoulos)
  /Subject  (Custom Computing: Assessed Coursework)
  /Keywords (custom,computing,coursework,imperial)
}
\begin{document}

\title{Custom Computing: Assessed Coursework}
\author{Ioannis Kassinopoulos}
\date{\today}
\maketitle
\section*{Question 1}
\textbf{Recurring engin
eering costs} are the costs that will occur in a repeating fashion during the production, usually involving fabriction.
These costs are usually descriped in a per unit form.
\\[0.25cm]
\textbf{Non-recurring engineering cost} is the one-time up-front cost 
for research, design, testing and development of a new product.
\\[0.25cm]
As we can see below, the minimum number of units that need to be sold for the ASIC implementation to be cost-effective is 1 million units.
\\[0.25cm]
$C_{FPGA} > C_{ASIC}$
$\Rightarrow \pounds 2 \times N_{units} > \pounds 10^6 + \pounds 1 \times N_{units} $ 
$\Rightarrow N_{units} > 10^6$

\section*{Question 2}
\subsection*{(a) Diagramatic and symbolic Simulation}
\subsubsection*{Diagram of circuit Q1}
\begin{figure}[!h]
\begin{center}
% Graphic for TeX using PGF
% Title: /home/yiannis/Diagram1.dia
% Creator: Dia v0.97.2
% CreationDate: Thu Feb 28 16:05:15 2013
% For: yiannis
% \usepackage{tikz}
% The following commands are not supported in PSTricks at present
% We define them conditionally, so when they are implemented,
% this pgf file will use them.
\ifx\du\undefined
  \newlength{\du}
\fi
\setlength{\du}{15\unitlength}
\begin{tikzpicture}[thick, scale=0.3]
\pgftransformxscale{1.000000}
\pgftransformyscale{-1.000000}
\definecolor{dialinecolor}{rgb}{0.000000, 0.000000, 0.000000}
\pgfsetstrokecolor{dialinecolor}
\definecolor{dialinecolor}{rgb}{1.000000, 1.000000, 1.000000}
\pgfsetfillcolor{dialinecolor}
\definecolor{dialinecolor}{rgb}{1.000000, 1.000000, 1.000000}
\pgfsetfillcolor{dialinecolor}
\fill (2.541048\du,-117.295272\du)--(2.541048\du,-110.044645\du)--(9.801177\du,-110.044645\du)--(9.801177\du,-117.295272\du)--cycle;
\pgfsetlinewidth{0.100000\du}
\pgfsetdash{}{0pt}
\pgfsetdash{}{0pt}
\pgfsetmiterjoin
\definecolor{dialinecolor}{rgb}{0.000000, 0.000000, 0.000000}
\pgfsetstrokecolor{dialinecolor}
\draw (2.541048\du,-117.295272\du)--(2.541048\du,-110.044645\du)--(9.801177\du,-110.044645\du)--(9.801177\du,-117.295272\du)--cycle;
% setfont left to latex
\definecolor{dialinecolor}{rgb}{0.000000, 0.000000, 0.000000}
\pgfsetstrokecolor{dialinecolor}
\node at (6.171113\du,-113.474958\du){ADD};
\definecolor{dialinecolor}{rgb}{1.000000, 1.000000, 1.000000}
\pgfsetfillcolor{dialinecolor}
\fill (2.535908\du,-127.308673\du)--(2.535908\du,-120.058046\du)--(9.796037\du,-120.058046\du)--(9.796037\du,-127.308673\du)--cycle;
\pgfsetlinewidth{0.100000\du}
\pgfsetdash{}{0pt}
\pgfsetdash{}{0pt}
\pgfsetmiterjoin
\definecolor{dialinecolor}{rgb}{0.000000, 0.000000, 0.000000}
\pgfsetstrokecolor{dialinecolor}
\draw (2.535908\du,-127.308673\du)--(2.535908\du,-120.058046\du)--(9.796037\du,-120.058046\du)--(9.796037\du,-127.308673\du)--cycle;
% setfont left to latex
\definecolor{dialinecolor}{rgb}{0.000000, 0.000000, 0.000000}
\pgfsetstrokecolor{dialinecolor}
\node at (6.165972\du,-123.488360\du){D-1};
% setfont left to latex
\definecolor{dialinecolor}{rgb}{0.000000, 0.000000, 0.000000}
\pgfsetstrokecolor{dialinecolor}
\node[anchor=west] at (5.384819\du,-123.499559\du){};
\pgfsetlinewidth{0.100000\du}
\pgfsetdash{}{0pt}
\pgfsetdash{}{0pt}
\pgfsetmiterjoin
\pgfsetbuttcap
{
\definecolor{dialinecolor}{rgb}{0.000000, 0.000000, 0.000000}
\pgfsetfillcolor{dialinecolor}
% was here!!!
{\pgfsetcornersarced{\pgfpoint{0.000000\du}{0.000000\du}}\definecolor{dialinecolor}{rgb}{0.000000, 0.000000, 0.000000}
\pgfsetstrokecolor{dialinecolor}
\draw (2.535908\du,-123.683360\du)--(-15.149771\du,-123.691849\du);
}}
\pgfsetlinewidth{0.100000\du}
\pgfsetdash{}{0pt}
\pgfsetdash{}{0pt}
\pgfsetbuttcap
{
\definecolor{dialinecolor}{rgb}{0.000000, 0.000000, 0.000000}
\pgfsetfillcolor{dialinecolor}
% was here!!!
\definecolor{dialinecolor}{rgb}{0.000000, 0.000000, 0.000000}
\pgfsetstrokecolor{dialinecolor}
\draw (2.541048\du,-111.857302\du)--(-15.149771\du,-111.836702\du);
}
\pgfsetlinewidth{0.100000\du}
\pgfsetdash{}{0pt}
\pgfsetdash{}{0pt}
\pgfsetmiterjoin
\pgfsetbuttcap
{
\definecolor{dialinecolor}{rgb}{0.000000, 0.000000, 0.000000}
\pgfsetfillcolor{dialinecolor}
% was here!!!
{\pgfsetcornersarced{\pgfpoint{0.000000\du}{0.000000\du}}\definecolor{dialinecolor}{rgb}{0.000000, 0.000000, 0.000000}
\pgfsetstrokecolor{dialinecolor}
\draw (2.541048\du,-115.482615\du)--(-11.106155\du,-115.482615\du)--(-11.106155\du,-123.645899\du)--(-11.152106\du,-123.645899\du);
}}
\pgfsetlinewidth{0.100000\du}
\pgfsetdash{}{0pt}
\pgfsetdash{}{0pt}
\pgfsetmiterjoin
\pgfsetbuttcap
{
\definecolor{dialinecolor}{rgb}{0.000000, 0.000000, 0.000000}
\pgfsetfillcolor{dialinecolor}
% was here!!!
{\pgfsetcornersarced{\pgfpoint{0.000000\du}{0.000000\du}}\definecolor{dialinecolor}{rgb}{0.000000, 0.000000, 0.000000}
\pgfsetstrokecolor{dialinecolor}
\draw (9.796037\du,-123.683360\du)--(15.177348\du,-123.737799\du);
}}
\pgfsetlinewidth{0.100000\du}
\pgfsetdash{}{0pt}
\pgfsetdash{}{0pt}
\pgfsetmiterjoin
\pgfsetbuttcap
{
\definecolor{dialinecolor}{rgb}{0.000000, 0.000000, 0.000000}
\pgfsetfillcolor{dialinecolor}
% was here!!!
{\pgfsetcornersarced{\pgfpoint{0.000000\du}{0.000000\du}}\definecolor{dialinecolor}{rgb}{0.000000, 0.000000, 0.000000}
\pgfsetstrokecolor{dialinecolor}
\draw (9.939028\du,-113.536859\du)--(15.085448\du,-113.536859\du);
}}
% setfont left to latex
\definecolor{dialinecolor}{rgb}{0.000000, 0.000000, 0.000000}
\pgfsetstrokecolor{dialinecolor}
\node[anchor=west] at (6.171113\du,-113.669958\du){};
% setfont left to latex
\definecolor{dialinecolor}{rgb}{0.000000, 0.000000, 0.000000}
\pgfsetstrokecolor{dialinecolor}
\node[anchor=west] at (6.165972\du,-123.683360\du){};
% setfont left to latex
\definecolor{dialinecolor}{rgb}{0.000000, 0.000000, 0.000000}
\pgfsetstrokecolor{dialinecolor}
\node[anchor=west] at (6.165972\du,-123.683360\du){};
\end{tikzpicture}

\caption{the circuit as derrived from Q1}
\end{center}
\end{figure}
\subsubsection*{Diagram of circuit P1}
\begin{figure}[!h]
\begin{center}
% Graphic for TeX using PGF
% Title: /home/yiannis/Desktop/cw-custom/diafiles/p1q1.dia
% Creator: Dia v0.97.2
% CreationDate: Thu Feb 28 16:59:01 2013
% For: yiannis
% \usepackage{tikz}
% The following commands are not supported in PSTricks at present
% We define them conditionally, so when they are implemented,
% this pgf file will use them.
\ifx\du\undefined
  \newlength{\du}
\fi
\setlength{\du}{15\unitlength}
\begin{tikzpicture}[thick, scale=0.26]
\pgftransformxscale{1.000000}
\pgftransformyscale{-1.000000}
\definecolor{dialinecolor}{rgb}{0.000000, 0.000000, 0.000000}
\pgfsetstrokecolor{dialinecolor}
\definecolor{dialinecolor}{rgb}{1.000000, 1.000000, 1.000000}
\pgfsetfillcolor{dialinecolor}
\definecolor{dialinecolor}{rgb}{1.000000, 1.000000, 1.000000}
\pgfsetfillcolor{dialinecolor}
\fill (14.240994\du,-115.995006\du)--(14.240994\du,-108.744379\du)--(21.501123\du,-108.744379\du)--(21.501123\du,-115.995006\du)--cycle;
\pgfsetlinewidth{0.100000\du}
\pgfsetdash{}{0pt}
\pgfsetdash{}{0pt}
\pgfsetmiterjoin
\definecolor{dialinecolor}{rgb}{0.000000, 0.000000, 0.000000}
\pgfsetstrokecolor{dialinecolor}
\draw (14.240994\du,-115.995006\du)--(14.240994\du,-108.744379\du)--(21.501123\du,-108.744379\du)--(21.501123\du,-115.995006\du)--cycle;
% setfont left to latex
\definecolor{dialinecolor}{rgb}{0.000000, 0.000000, 0.000000}
\pgfsetstrokecolor{dialinecolor}
\node at (17.871058\du,-112.197471\du){ADD};
\definecolor{dialinecolor}{rgb}{1.000000, 1.000000, 1.000000}
\pgfsetfillcolor{dialinecolor}
\fill (14.235854\du,-126.009006\du)--(14.235854\du,-118.758379\du)--(21.495983\du,-118.758379\du)--(21.495983\du,-126.009006\du)--cycle;
\pgfsetlinewidth{0.100000\du}
\pgfsetdash{}{0pt}
\pgfsetdash{}{0pt}
\pgfsetmiterjoin
\definecolor{dialinecolor}{rgb}{0.000000, 0.000000, 0.000000}
\pgfsetstrokecolor{dialinecolor}
\draw (14.235854\du,-126.009006\du)--(14.235854\du,-118.758379\du)--(21.495983\du,-118.758379\du)--(21.495983\du,-126.009006\du)--cycle;
% setfont left to latex
\definecolor{dialinecolor}{rgb}{0.000000, 0.000000, 0.000000}
\pgfsetstrokecolor{dialinecolor}
\node at (17.865918\du,-122.211471\du){D-1};
% setfont left to latex
\definecolor{dialinecolor}{rgb}{0.000000, 0.000000, 0.000000}
\pgfsetstrokecolor{dialinecolor}
\node[anchor=west] at (17.084764\du,-122.200006\du){};
\pgfsetlinewidth{0.100000\du}
\pgfsetdash{}{0pt}
\pgfsetdash{}{0pt}
\pgfsetmiterjoin
\pgfsetbuttcap
{
\definecolor{dialinecolor}{rgb}{0.000000, 0.000000, 0.000000}
\pgfsetfillcolor{dialinecolor}
% was here!!!
\pgfsetarrowsend{to}
{\pgfsetcornersarced{\pgfpoint{0.000000\du}{0.000000\du}}\definecolor{dialinecolor}{rgb}{0.000000, 0.000000, 0.000000}
\pgfsetstrokecolor{dialinecolor}
\draw (14.235854\du,-122.383693\du)--(-3.449856\du,-122.392006\du);
}}
\pgfsetlinewidth{0.100000\du}
\pgfsetdash{}{0pt}
\pgfsetdash{}{0pt}
\pgfsetbuttcap
{
\definecolor{dialinecolor}{rgb}{0.000000, 0.000000, 0.000000}
\pgfsetfillcolor{dialinecolor}
% was here!!!
\pgfsetarrowsstart{to}
\definecolor{dialinecolor}{rgb}{0.000000, 0.000000, 0.000000}
\pgfsetstrokecolor{dialinecolor}
\draw (14.240994\du,-110.557036\du)--(-3.449856\du,-110.537006\du);
}
\pgfsetlinewidth{0.100000\du}
\pgfsetdash{}{0pt}
\pgfsetdash{}{0pt}
\pgfsetmiterjoin
\pgfsetbuttcap
{
\definecolor{dialinecolor}{rgb}{0.000000, 0.000000, 0.000000}
\pgfsetfillcolor{dialinecolor}
% was here!!!
\pgfsetarrowsstart{to}
{\pgfsetcornersarced{\pgfpoint{0.000000\du}{0.000000\du}}\definecolor{dialinecolor}{rgb}{0.000000, 0.000000, 0.000000}
\pgfsetstrokecolor{dialinecolor}
\draw (14.240994\du,-114.182349\du)--(0.593744\du,-114.182349\du)--(0.593744\du,-122.346006\du)--(0.547844\du,-122.346006\du);
}}
% setfont left to latex
\definecolor{dialinecolor}{rgb}{0.000000, 0.000000, 0.000000}
\pgfsetstrokecolor{dialinecolor}
\node[anchor=west] at (17.871058\du,-112.369693\du){};
% setfont left to latex
\definecolor{dialinecolor}{rgb}{0.000000, 0.000000, 0.000000}
\pgfsetstrokecolor{dialinecolor}
\node[anchor=west] at (17.865918\du,-122.383693\du){};
% setfont left to latex
\definecolor{dialinecolor}{rgb}{0.000000, 0.000000, 0.000000}
\pgfsetstrokecolor{dialinecolor}
\node[anchor=west] at (17.865918\du,-122.383693\du){};
\definecolor{dialinecolor}{rgb}{1.000000, 1.000000, 1.000000}
\pgfsetfillcolor{dialinecolor}
\fill (39.080876\du,-115.993845\du)--(39.080876\du,-108.743218\du)--(46.341004\du,-108.743218\du)--(46.341004\du,-115.993845\du)--cycle;
\pgfsetlinewidth{0.100000\du}
\pgfsetdash{}{0pt}
\pgfsetdash{}{0pt}
\pgfsetmiterjoin
\definecolor{dialinecolor}{rgb}{0.000000, 0.000000, 0.000000}
\pgfsetstrokecolor{dialinecolor}
\draw (39.080876\du,-115.993845\du)--(39.080876\du,-108.743218\du)--(46.341004\du,-108.743218\du)--(46.341004\du,-115.993845\du)--cycle;
% setfont left to latex
\definecolor{dialinecolor}{rgb}{0.000000, 0.000000, 0.000000}
\pgfsetstrokecolor{dialinecolor}
\node at (42.710940\du,-112.196309\du){ADD};
\definecolor{dialinecolor}{rgb}{1.000000, 1.000000, 1.000000}
\pgfsetfillcolor{dialinecolor}
\fill (39.075736\du,-126.007845\du)--(39.075736\du,-118.757218\du)--(46.335864\du,-118.757218\du)--(46.335864\du,-126.007845\du)--cycle;
\pgfsetlinewidth{0.100000\du}
\pgfsetdash{}{0pt}
\pgfsetdash{}{0pt}
\pgfsetmiterjoin
\definecolor{dialinecolor}{rgb}{0.000000, 0.000000, 0.000000}
\pgfsetstrokecolor{dialinecolor}
\draw (39.075736\du,-126.007845\du)--(39.075736\du,-118.757218\du)--(46.335864\du,-118.757218\du)--(46.335864\du,-126.007845\du)--cycle;
% setfont left to latex
\definecolor{dialinecolor}{rgb}{0.000000, 0.000000, 0.000000}
\pgfsetstrokecolor{dialinecolor}
\node at (42.705800\du,-122.210309\du){D-1};
% setfont left to latex
\definecolor{dialinecolor}{rgb}{0.000000, 0.000000, 0.000000}
\pgfsetstrokecolor{dialinecolor}
\node[anchor=west] at (41.924646\du,-122.198845\du){};
\pgfsetlinewidth{0.100000\du}
\pgfsetdash{}{0pt}
\pgfsetdash{}{0pt}
\pgfsetmiterjoin
\pgfsetbuttcap
{
\definecolor{dialinecolor}{rgb}{0.000000, 0.000000, 0.000000}
\pgfsetfillcolor{dialinecolor}
% was here!!!
\pgfsetarrowsend{to}
{\pgfsetcornersarced{\pgfpoint{0.000000\du}{0.000000\du}}\definecolor{dialinecolor}{rgb}{0.000000, 0.000000, 0.000000}
\pgfsetstrokecolor{dialinecolor}
\draw (39.075736\du,-122.382532\du)--(21.495983\du,-122.383693\du);
}}
\pgfsetlinewidth{0.100000\du}
\pgfsetdash{}{0pt}
\pgfsetdash{}{0pt}
\pgfsetbuttcap
{
\definecolor{dialinecolor}{rgb}{0.000000, 0.000000, 0.000000}
\pgfsetfillcolor{dialinecolor}
% was here!!!
\pgfsetarrowsstart{to}
\definecolor{dialinecolor}{rgb}{0.000000, 0.000000, 0.000000}
\pgfsetstrokecolor{dialinecolor}
\draw (39.080876\du,-110.555875\du)--(21.501123\du,-110.557036\du);
}
\pgfsetlinewidth{0.100000\du}
\pgfsetdash{}{0pt}
\pgfsetdash{}{0pt}
\pgfsetmiterjoin
\pgfsetbuttcap
{
\definecolor{dialinecolor}{rgb}{0.000000, 0.000000, 0.000000}
\pgfsetfillcolor{dialinecolor}
% was here!!!
\pgfsetarrowsstart{to}
{\pgfsetcornersarced{\pgfpoint{0.000000\du}{0.000000\du}}\definecolor{dialinecolor}{rgb}{0.000000, 0.000000, 0.000000}
\pgfsetstrokecolor{dialinecolor}
\draw (39.080876\du,-114.181188\du)--(25.433626\du,-114.181188\du)--(25.433626\du,-122.344845\du)--(25.387726\du,-122.344845\du);
}}
% setfont left to latex
\definecolor{dialinecolor}{rgb}{0.000000, 0.000000, 0.000000}
\pgfsetstrokecolor{dialinecolor}
\node[anchor=west] at (42.710940\du,-112.368532\du){};
% setfont left to latex
\definecolor{dialinecolor}{rgb}{0.000000, 0.000000, 0.000000}
\pgfsetstrokecolor{dialinecolor}
\node[anchor=west] at (42.705800\du,-122.382532\du){};
% setfont left to latex
\definecolor{dialinecolor}{rgb}{0.000000, 0.000000, 0.000000}
\pgfsetstrokecolor{dialinecolor}
\node[anchor=west] at (42.705800\du,-122.382532\du){};
\definecolor{dialinecolor}{rgb}{1.000000, 1.000000, 1.000000}
\pgfsetfillcolor{dialinecolor}
\fill (63.880875\du,-115.984880\du)--(63.880875\du,-108.734253\du)--(71.141004\du,-108.734253\du)--(71.141004\du,-115.984880\du)--cycle;
\pgfsetlinewidth{0.100000\du}
\pgfsetdash{}{0pt}
\pgfsetdash{}{0pt}
\pgfsetmiterjoin
\definecolor{dialinecolor}{rgb}{0.000000, 0.000000, 0.000000}
\pgfsetstrokecolor{dialinecolor}
\draw (63.880875\du,-115.984880\du)--(63.880875\du,-108.734253\du)--(71.141004\du,-108.734253\du)--(71.141004\du,-115.984880\du)--cycle;
% setfont left to latex
\definecolor{dialinecolor}{rgb}{0.000000, 0.000000, 0.000000}
\pgfsetstrokecolor{dialinecolor}
\node at (67.510940\du,-112.187345\du){ADD};
\definecolor{dialinecolor}{rgb}{1.000000, 1.000000, 1.000000}
\pgfsetfillcolor{dialinecolor}
\fill (63.875735\du,-125.998880\du)--(63.875735\du,-118.748253\du)--(71.135864\du,-118.748253\du)--(71.135864\du,-125.998880\du)--cycle;
\pgfsetlinewidth{0.100000\du}
\pgfsetdash{}{0pt}
\pgfsetdash{}{0pt}
\pgfsetmiterjoin
\definecolor{dialinecolor}{rgb}{0.000000, 0.000000, 0.000000}
\pgfsetstrokecolor{dialinecolor}
\draw (63.875735\du,-125.998880\du)--(63.875735\du,-118.748253\du)--(71.135864\du,-118.748253\du)--(71.135864\du,-125.998880\du)--cycle;
% setfont left to latex
\definecolor{dialinecolor}{rgb}{0.000000, 0.000000, 0.000000}
\pgfsetstrokecolor{dialinecolor}
\node at (67.505800\du,-122.201345\du){D-1};
% setfont left to latex
\definecolor{dialinecolor}{rgb}{0.000000, 0.000000, 0.000000}
\pgfsetstrokecolor{dialinecolor}
\node[anchor=west] at (66.724645\du,-122.189880\du){};
\pgfsetlinewidth{0.100000\du}
\pgfsetdash{}{0pt}
\pgfsetdash{}{0pt}
\pgfsetmiterjoin
\pgfsetbuttcap
{
\definecolor{dialinecolor}{rgb}{0.000000, 0.000000, 0.000000}
\pgfsetfillcolor{dialinecolor}
% was here!!!
\pgfsetarrowsend{to}
{\pgfsetcornersarced{\pgfpoint{0.000000\du}{0.000000\du}}\definecolor{dialinecolor}{rgb}{0.000000, 0.000000, 0.000000}
\pgfsetstrokecolor{dialinecolor}
\draw (63.875735\du,-122.373567\du)--(46.335864\du,-122.382532\du);
}}
\pgfsetlinewidth{0.100000\du}
\pgfsetdash{}{0pt}
\pgfsetdash{}{0pt}
\pgfsetbuttcap
{
\definecolor{dialinecolor}{rgb}{0.000000, 0.000000, 0.000000}
\pgfsetfillcolor{dialinecolor}
% was here!!!
\pgfsetarrowsstart{to}
\definecolor{dialinecolor}{rgb}{0.000000, 0.000000, 0.000000}
\pgfsetstrokecolor{dialinecolor}
\draw (63.880875\du,-110.546910\du)--(46.341004\du,-110.555875\du);
}
\pgfsetlinewidth{0.100000\du}
\pgfsetdash{}{0pt}
\pgfsetdash{}{0pt}
\pgfsetmiterjoin
\pgfsetbuttcap
{
\definecolor{dialinecolor}{rgb}{0.000000, 0.000000, 0.000000}
\pgfsetfillcolor{dialinecolor}
% was here!!!
\pgfsetarrowsstart{to}
{\pgfsetcornersarced{\pgfpoint{0.000000\du}{0.000000\du}}\definecolor{dialinecolor}{rgb}{0.000000, 0.000000, 0.000000}
\pgfsetstrokecolor{dialinecolor}
\draw (63.880875\du,-114.172224\du)--(50.233625\du,-114.172224\du)--(50.233625\du,-122.335880\du)--(50.187725\du,-122.335880\du);
}}
% setfont left to latex
\definecolor{dialinecolor}{rgb}{0.000000, 0.000000, 0.000000}
\pgfsetstrokecolor{dialinecolor}
\node[anchor=west] at (67.510940\du,-112.359567\du){};
% setfont left to latex
\definecolor{dialinecolor}{rgb}{0.000000, 0.000000, 0.000000}
\pgfsetstrokecolor{dialinecolor}
\node[anchor=west] at (67.505800\du,-122.373567\du){};
% setfont left to latex
\definecolor{dialinecolor}{rgb}{0.000000, 0.000000, 0.000000}
\pgfsetstrokecolor{dialinecolor}
\node[anchor=west] at (67.505800\du,-122.373567\du){};
\definecolor{dialinecolor}{rgb}{1.000000, 1.000000, 1.000000}
\pgfsetfillcolor{dialinecolor}
\fill (88.728478\du,-115.987915\du)--(88.728478\du,-108.737289\du)--(95.988607\du,-108.737289\du)--(95.988607\du,-115.987915\du)--cycle;
\pgfsetlinewidth{0.100000\du}
\pgfsetdash{}{0pt}
\pgfsetdash{}{0pt}
\pgfsetmiterjoin
\definecolor{dialinecolor}{rgb}{0.000000, 0.000000, 0.000000}
\pgfsetstrokecolor{dialinecolor}
\draw (88.728478\du,-115.987915\du)--(88.728478\du,-108.737289\du)--(95.988607\du,-108.737289\du)--(95.988607\du,-115.987915\du)--cycle;
% setfont left to latex
\definecolor{dialinecolor}{rgb}{0.000000, 0.000000, 0.000000}
\pgfsetstrokecolor{dialinecolor}
\node at (92.358542\du,-112.190380\du){ADD};
\definecolor{dialinecolor}{rgb}{1.000000, 1.000000, 1.000000}
\pgfsetfillcolor{dialinecolor}
\fill (88.723338\du,-126.001915\du)--(88.723338\du,-118.751289\du)--(95.983467\du,-118.751289\du)--(95.983467\du,-126.001915\du)--cycle;
\pgfsetlinewidth{0.100000\du}
\pgfsetdash{}{0pt}
\pgfsetdash{}{0pt}
\pgfsetmiterjoin
\definecolor{dialinecolor}{rgb}{0.000000, 0.000000, 0.000000}
\pgfsetstrokecolor{dialinecolor}
\draw (88.723338\du,-126.001915\du)--(88.723338\du,-118.751289\du)--(95.983467\du,-118.751289\du)--(95.983467\du,-126.001915\du)--cycle;
% setfont left to latex
\definecolor{dialinecolor}{rgb}{0.000000, 0.000000, 0.000000}
\pgfsetstrokecolor{dialinecolor}
\node at (92.353402\du,-122.204380\du){D-1};
% setfont left to latex
\definecolor{dialinecolor}{rgb}{0.000000, 0.000000, 0.000000}
\pgfsetstrokecolor{dialinecolor}
\node[anchor=west] at (91.572248\du,-122.192915\du){};
\pgfsetlinewidth{0.100000\du}
\pgfsetdash{}{0pt}
\pgfsetdash{}{0pt}
\pgfsetmiterjoin
\pgfsetbuttcap
{
\definecolor{dialinecolor}{rgb}{0.000000, 0.000000, 0.000000}
\pgfsetfillcolor{dialinecolor}
% was here!!!
\pgfsetarrowsend{to}
{\pgfsetcornersarced{\pgfpoint{0.000000\du}{0.000000\du}}\definecolor{dialinecolor}{rgb}{0.000000, 0.000000, 0.000000}
\pgfsetstrokecolor{dialinecolor}
\draw (88.723338\du,-122.376602\du)--(71.135864\du,-122.373567\du);
}}
\pgfsetlinewidth{0.100000\du}
\pgfsetdash{}{0pt}
\pgfsetdash{}{0pt}
\pgfsetbuttcap
{
\definecolor{dialinecolor}{rgb}{0.000000, 0.000000, 0.000000}
\pgfsetfillcolor{dialinecolor}
% was here!!!
\pgfsetarrowsstart{to}
\definecolor{dialinecolor}{rgb}{0.000000, 0.000000, 0.000000}
\pgfsetstrokecolor{dialinecolor}
\draw (88.728478\du,-110.549945\du)--(71.141004\du,-110.546910\du);
}
\pgfsetlinewidth{0.100000\du}
\pgfsetdash{}{0pt}
\pgfsetdash{}{0pt}
\pgfsetmiterjoin
\pgfsetbuttcap
{
\definecolor{dialinecolor}{rgb}{0.000000, 0.000000, 0.000000}
\pgfsetfillcolor{dialinecolor}
% was here!!!
\pgfsetarrowsstart{to}
{\pgfsetcornersarced{\pgfpoint{0.000000\du}{0.000000\du}}\definecolor{dialinecolor}{rgb}{0.000000, 0.000000, 0.000000}
\pgfsetstrokecolor{dialinecolor}
\draw (88.728478\du,-114.175259\du)--(75.081228\du,-114.175259\du)--(75.081228\du,-122.338915\du)--(75.035328\du,-122.338915\du);
}}
% setfont left to latex
\definecolor{dialinecolor}{rgb}{0.000000, 0.000000, 0.000000}
\pgfsetstrokecolor{dialinecolor}
\node[anchor=west] at (92.358542\du,-112.362602\du){};
% setfont left to latex
\definecolor{dialinecolor}{rgb}{0.000000, 0.000000, 0.000000}
\pgfsetstrokecolor{dialinecolor}
\node[anchor=west] at (92.353402\du,-122.376602\du){};
% setfont left to latex
\definecolor{dialinecolor}{rgb}{0.000000, 0.000000, 0.000000}
\pgfsetstrokecolor{dialinecolor}
\node[anchor=west] at (92.353402\du,-122.376602\du){};
\pgfsetlinewidth{0.100000\du}
\pgfsetdash{}{0pt}
\pgfsetdash{}{0pt}
\pgfsetbuttcap
{
\definecolor{dialinecolor}{rgb}{0.000000, 0.000000, 0.000000}
\pgfsetfillcolor{dialinecolor}
% was here!!!
\pgfsetarrowsend{to}
\definecolor{dialinecolor}{rgb}{0.000000, 0.000000, 0.000000}
\pgfsetstrokecolor{dialinecolor}
\pgfpathmoveto{\pgfpoint{95.988211\du}{-112.362599\du}}
\pgfpatharc{90}{-89}{5.007175\du and 5.007175\du}
\pgfusepath{stroke}
}
\pgfsetlinewidth{0.100000\du}
\pgfsetdash{}{0pt}
\pgfsetdash{}{0pt}
\pgfsetbuttcap
{
\definecolor{dialinecolor}{rgb}{0.000000, 0.000000, 0.000000}
\pgfsetfillcolor{dialinecolor}
% was here!!!
\pgfsetarrowsend{to}
\definecolor{dialinecolor}{rgb}{0.000000, 0.000000, 0.000000}
\pgfsetstrokecolor{dialinecolor}
\draw (100.951252\du,-117.555925\du)--(106.351178\du,-117.555925\du);
}
\end{tikzpicture}

\caption{the circuit as derrived from P1}
\end{center}
\end{figure}
\subsubsection*{Simulation}
The source code of the simulation (uninitialized delay) is the following:
\begin{Verbatim}[commandchars=\\\{\}]
INCLUDE "prelude.rby".
P1 n = Q1^n; fork^\mytilde1 .
Q1 = snd fork; rsh; [add,D^\mytilde1].
current = P1 4.
\end{Verbatim}
The circuit representation:
\begin{Verbatim}[commandchars=\\\{\}]
   Name           Domain             Range
   ----------------------------------------
   D ?            .1                 .2
   D ?            .3                 .1
   D ?            .4                 .3
   D ?            .5                 .4
   ----------------------------------------
   add            <.6,.2>            .7
   ----------------------------------------
   add            <.7,.1>            .8
   ----------------------------------------
   add            <.8,.3>            .9
   ----------------------------------------
   add            <.9,.4>            .5
   ----------------------------------------

 Directions -  <in,out> \mytilde out

 Wiring -  <.6,.2> \mytilde .5

 Inputs -  .6
\end{Verbatim}

The source code of the simulation (initialized delay with 0) is the following:
\begin{Verbatim}[commandchars=\\\{\}]
INCLUDE "prelude.rby".
P1 n = Q1^n; fork^\mytilde1 .
Q1 = snd fork; rsh; [add,DI 0^\mytilde1].
current = P1 4.
\end{Verbatim}
The circuit representation:
\begin{Verbatim}
    Name           Domain             Range
   ----------------------------------------
   D 0            .1                 .2
   D 0            .3                 .1
   D 0            .4                 .3
   D 0            .5                 .4
   ----------------------------------------
   add            <.6,.2>            .7
   ----------------------------------------
   add            <.7,.1>            .8
   ----------------------------------------
   add            <.8,.3>            .9
   ----------------------------------------
   add            <.9,.4>            .5
   ----------------------------------------

 Directions -  <in,out> ~ out

 Wiring -  <.6,.2> ~ .5

 Inputs -  .6

\end{Verbatim}
The simulation output (for 4 cycles) can be found in the included zip file.
\section*{Question 3}
\subsection*{(a) Proof by induction}
In order to show that $[P,Q]^{n};R = R;Q^n$ for $n>0$, we first have to show that it is $True$ for n = 1.
\\[0.5cm]
Base case: $[P,Q]^{1};R=R;Q^1$
\\[0.5cm]
This is intuitively shown to be true by the given assumption $[P,Q]^{n};R$ which is equivalent.
\\[0.5cm]
Assuming that it is also true for  $n = k > 0$ 
\\[0.5cm]
$[P,Q]^{k};R = R;Q^k$
\\[0.5cm]
We need to show that the same is true for $n = k+1$ 
\\[0.5cm]
$[P,Q]^{k+1};R $
\\[0.25cm]
$= [P,Q]^k;[P,Q];R$
\\[0.25cm]
$= [P,Q]^k;R;Q$
\\[0.25cm]
$= R;Q^k;Q$
\\[0.25cm]
$= R;Q^{k+1}$
\\[0.5cm]
So by induction we have proved that if we know $[P,Q];R = R;Q$ to be $True$, for $n>0$:
\\[0.5cm]
$[P,Q]^{n};R = R;Q^n$
is also $True$

\section*{Question 4}
\end{document}
