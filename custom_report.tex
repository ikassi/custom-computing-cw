\documentclass[a4paper,10pt]{article}
%\documentclass[a4paper,10pt]{scrartcl}

\usepackage[utf8]{inputenc}
\usepackage[margin=1in]{geometry}

\pdfinfo{%
  /Title    (Custom Computing: Assessed Coursework)
  /Author   (Ioannis Kassinopoulos)
  /Creator  (Ioannis Kassinopoulos)
  /Producer (Ioannis Kassinopoulos)
  /Subject  (Custom Computing: Assessed Coursework)
  /Keywords (custom,computing,coursework,imperial)
}
\begin{document}

\title{Custom Computing: Assessed Coursework}
\author{Ioannis Kassinopoulos}
\date{\today}
\maketitle
\section*{Question 1}
\textbf{Recurring engineering costs} are the costs that will occur in a repeating fashion during the production, usually involving fabriction.
These costs are usually descriped in a per unit form.
\\[0.25cm]
\textbf{Non-recurring engineering cost} is the one-time up-front cost 
for research, design, testing and development of a new product.
\\[0.25cm]
As we can see below, the minimum number of units that need to be sold for the ASIC implementation to be cost-effective is 1 million units.
\\[0.25cm]
$C_{FPGA} > C_{ASIC}$
$\Rightarrow \pounds 2 \times N_{units} > \pounds 10^6 + \pounds 1 \times N_{units} $ 
$\Rightarrow N_{units} > 10^6$

\section*{Question 2}
\section*{Question 3}
In order to show that $[P,Q]^{n};R = R;Q^n$ for $n>0$, we first have to show that it is $True$ for n = 1.
\\[0.5cm]
Base case: $[P,Q]^{1};R=R;Q^1$
\\[0.5cm]
This is intuitively shown to be true by the given assumption $[P,Q]^{n};R$ which is equivalent.
\\[0.5cm]
Assuming that it is also true for  $n = k > 0$ 
\\[0.5cm]
$[P,Q]^{k};R = R;Q^k$
\\[0.5cm]
We need to show that the same is true for $n = k+1$ 
\\[0.5cm]
$[P,Q]^{k+1};R $
\\[0.25cm]
$= [P,Q]^k;[P,Q];R$
\\[0.25cm]
$= [P,Q]^k;R;Q$
\\[0.25cm]
$= R;Q^k;Q$
\\[0.25cm]
$= R;Q^{k+1}$
\\[0.5cm]
So by induction we have proved that if we know $[P,Q];R = R;Q$ to be $True$, for $n>0$:
\\[0.5cm]
$[P,Q]^{n};R = R;Q^n$
is also $True$

\section*{Question 4}
\end{document}
